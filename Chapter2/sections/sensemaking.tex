\section{Sensemaking}
\emph{Sensemaking} reflects how we make sense of the world so that we can act in it~\cite{Snowden2005}. Sensemaking has been studied in different contexts such as information science~\cite{Dervin1983}, human-computer interaction~\cite{Russell1993}, organizational studies~\cite{Weick1995}, and intelligence analysis~\cite{Pirolli2005,Klein2003}. In this section, we review the sensemaking concepts discussed in these contexts, with an emphasis on the last two sensemaking models that have been highly applied in the visualization community. A recent and comprehensive review of sensemaking can be found in the article by Maitlis and Christianson~\cite{Maitlis2014}.

\subsection{Gap-Bridging Metaphor}
Dervin develops a sensemaking theory focusing on information seeking and use behaviors~\cite{Dervin1983}. It underlies the cognitive gap that individuals experience when attempting to make sense of observed data. \autoref{fig:lr-dervin} summarizes this \emph{gap-bridging} metaphor. The theory assumes that people moves through time-space in some particular context and situation. Sensemaking starts when they encounter a gap that needs to overcome such as something is unclear or confused. To bridge the gap, they may seek and use information from a variety of sources such as documents, media and other people. These sources are evaluated based on relevant attributes to assess their usefulness. 

\begin{figure}[!htb]
	\centering
	\includegraphics[width=\columnwidth]{dervin}
	\caption{The gap-bridging metaphor of sensemaking. People encounter gaps when moving through time-space, then seek for information, evaluate and use it to bridge the gaps. \is{Dervin2012}}
	\label{fig:lr-dervin}
\end{figure}

Dervin also implements the theory into a set of questions that can be used in interview to understand sensemaking within a context~\cite{Dervin1983}. The questions elaborate all parts of the model, aiming to establish an understanding on the situation (\emph{What happened?}), the gap (\emph{What did you struggle with?}), the bridge (\emph{What idea did you come to?}) and the outcome (\emph{How did that help?}).

\subsection{Learning Loop Complex}
In the context of human-computer interaction, Russell~et~al.~\cite{Russell1993} defines sensemaking as the process of searching for a representation and encoding data in that representation to answer task-specific questions. That cyclic process is called the \emph{learning loop complex} as illustrated in \autoref{fig:lr-russell}. First, the sensemaker searches for a representation to capture salient features of the data (\emph{Generation Loop}). During sensemaking, new information is sought and encoded into this representation (\emph{Data Coverage Loop}). The data unfit to the representation (\emph{residue}) requires the sensemaker to adjusts and produces a more suitable one. This entire learning loop complex is guided by the task with an aim to reduce its cost.

\begin{figure}[!htb]
	\centering
	\includegraphics{russell}
	\caption{The learning loop complex theory of sensemaking. It consists of three iterative loops: searching for a good representation, encoding data to the representation, and adjusting the representation for a better data coverage. \is{Russell1993}}
	\label{fig:lr-russell}
\end{figure}

\subsection{Sensemaking in Organizations}
Different from Dervin and Russell who study sensemaking for individuals, Weick focuses on sensemaking at an organization level~\cite{Weick1995}. He proposes that sensemaking consists of these seven following properties.

\begin{enumerate}
	\item \emph{Grounded in identity construction}. Who people think they are, both individually and collectively, affect what they interpret and act.
	\item \emph{Retrospective}. People look back and make sense from what they have said and what they have done before.
	\item \emph{Enactive of sensible environments}. People make sense and contribute to the environments during their sensemaking processes.
	\item \emph{Social}. This is an inherent property of sensemaking in organization where people interact and socialize with others, and also are influenced by others.
	\item \emph{Ongoing}. Sensemaking is a continuous flow because the world and our understanding about the word are constantly changing.
	\item \emph{Focused on and by extracted cues}. Cues are things from the context that people have attention to and may use them to guide further exploration and assessment of the sensemaking problem.
	\item \emph{Driven by plausibility rather than accuracy}. Sensemaking is about plausibility and sufficiency rather than accuracy and completeness. People tend to stop searching when they find an acceptable solution.
\end{enumerate}

\subsection{A Process Model of Sensemaking}
\label{sub:rv-pcm}
Pirolli and Card~\cite{Pirolli2005} describe sensemaking as an iterative process that gradually transforms raw data into rational knowledge. The process includes two sets of activities: one that cycles around finding relevant information, and another that cycles around making sense of that information, with plenty of interaction between them. They map to the \emph{foraging loop} and the \emph{sensemaking loop} respectively, as shown in \autoref{fig:lr-pirolli-card-model}. The sensemaking process can progress upward (from data to knowledge) or downward (from knowledge to data). The steps in the \emph{bottom-up} process are summarized as follows.

\begin{figure}[!htb]
	\centering
	\includegraphics[width=\columnwidth]{pirolli-card-model}
	\caption{A notional model of sensemaking. The sub-processes (numbered circles) and their data input/output (numbered rectangles) are arranged in a two-dimensional space, in which the horizontal axis represents the degree of effort from users, and the vertical axis represents the degree of structure in information representation. \is{Pirolli2005}}
	\label{fig:lr-pirolli-card-model}
\end{figure}

\begin{itemize}
	\item \emph{Search and filter}. External data sources, such as classified databases or the web, are searched and filtered to retrieve relevant documents to the task.
	\item \emph{Read and extract}. These documents are examined to extract pieces of information that may be used as evidence later.
	\item \emph{Schematize}.  The collected information is organized in a way that aids the analysis. This may be executed in user mind, using paper and pen, or with a complex computer-based system.
	\item \emph{Build case}. Multiple hypotheses are generated; evidence are marshaled to support or disconfirm them.
	\item \emph{Tell story}. Discovered cases are presented to some audience.
\end{itemize}

In this model, \emph{schematization} plays an important role in converting raw evidence to rational explanations, bridging the foraging and sensemaking loops. A study by Kang, Görg and John Stasko~\cite{Kang2011} agrees with this observation. In their study, all the participants who performed the sensemaking task well spent considerable time and effort in organizing their collected information. Their organizational schemes were flexible: a \emph{timeline} of related events, a \emph{map} connecting locations that a person has been to, and a \emph{diagram} showing relationships among suspicious targets.

\subsection{Data--Frame Model}
\label{sub:rv-dfm}
Klein et al.~\cite{Klein2003} propose a sensemaking model that centers around \emph{data} and \emph{frame}. Data is the information that a person receives or searches for, and frame is the mental structure that organizes and explains the relationship of such data. For instance, a frame can be a \emph{story}, explaining the chronology of events and the causal relationships between
them; or a \emph{map}, showing where the events take place and the routes between them. Sensemaking is considered as a deliberate effort to understand an event, starting when a person realizes a gap of their current understanding of that event. Klein and his associates describe seven activities involved in sensemaking and are summarized in \autoref{fig:lr-data-frame-model}.

\begin{figure}[!htb]
	\centering
	\includegraphics[width=\columnwidth]{data-frame-model}
	\caption{The data--frame model of sensemaking. It describes a set of interconnected sensemaking activities centering around data and frame -- the explanatory structure of data. \is{Klein2003}}
	\label{fig:lr-data-frame-model}
\end{figure}

\begin{itemize}
	\item \emph{Connect data and a frame}. A person recognizes relevant pieces of data and constructs an initial frame to explain them. The frame then helps the person to filter and search for new data.
	\item \emph{Elaborate the frame}. As more is learned about the situation, the frame becomes more elaborate with new data and new relationships. 
	\item \emph{Question the frame}. It happens when a person encounters data that is inconsistent with the existing frame. At this point, the person may be unsure that the frame is incorrect, or the inconsistent data is inaccurate.
	\item \emph{Preserve the frame}. A person may consider the severity of the inconsistent data, justify why it mismatches the frame, and ignore it.
	\item \emph{Compare multiple frames}. Depending on experience, a person may think of alternative frames explaining the same set of data. These frames need to be compared to select the most likely one.
	\item \emph{Reframe}. When encountering inconsistent and contrary data, the person may need to find a replacement that can explain all data. Considering discarded data and/or reinterpreting data could facilitate this activity.
	\item \emph{Seek a new frame}. A person may deliberately search for a new frame when encountering plenty of conflicted data. One or two key data elements may serve as \emph{anchors} to help the person to elicit another frame.
\end{itemize}

The Pirolli and Card's model describes a step-by-step process of sensemaking, in which the analyst collects relevant data and eventually transposes it into rational answers. However, the various sensemaking activities in the Data--Frame model may explain the strategies used by the analyst more comprehensively.