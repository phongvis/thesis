\section{Information Design Principles}
\begin{itemize}
	\item encoding channel ranking (chap 5, 6 Munzner book)
%	\item some of 'rules of thumbs' in her book as well
	\item Focus+Context
	\item Gestalt Principles
	\item Principle of Visual Affordance
	\item Tufte: ink-ratio
	\item Ben Shneiderman’s influential mantra of Overview First, Zoom and Filter, Details on Demand
\end{itemize}



\subsection{Color}

\subsubsection {Color Spaces}
RGB and HSL

\subsubsection {Colormaps}

categorical
sequential
diverging

%pre-attentive

%Proximity-Compatibility Principle


%Ecological Interface Design
%Representation of Functional Relationships

Several interface design guidelines: Schneiderman (eight golden rules), Norman (design principles) and Nielsen’s (ten usability heuristics)

%https://en.wikipedia.org/wiki/Human%E2%80%93computer_interaction#Display_designs
%https://www.interaction-design.org/literature/book/the-glossary-of-human-computer-interaction/affordances


%Working memory becomes an important factor as it organise and process information
%when learning occurs (Global, 2000). Here information is sorted and organised into
%relevant schema. Schemas can be referred to as models or hypothetical structures
%that organises once knowledge of the world.
%
%perception for design, Colin Ware book
%


