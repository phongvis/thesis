\section{Related Work}
\label{sec:relatedwork}

%\subsection{Timeline Visualization}
% Timeline visualization
%Timeline is one of the earliest visualizations. Back in 1765, Joseph Priestley created the Chart of Biography about two thousand famous people from 1200 B.C to 1750 A.D~\cite{Priestley1765} -- one of the oldest documented timeline. Along a horizontal timeline, Priestley used bars to depict lifespans and dots to illustrate the uncertain birth and dead dates. This remains one of the most popular visual metaphors for timeline visualization: time is represented as a horizontal axis; an event is plotted as a point if it is an instant or a bar spans its interval otherwise. Geographical information can also be included in timeline visualizations as in the classic information visualization example of Napoleon's March on Moscow in 1812-1813 by Charles Joseph Minard~\cite{Minard1869}. 

In this section, we consider related work on representing temporal data and producing timelines, before also examining larger visual analytic systems that include timelines in their feature set.

\subsection{Timeline Visualizations}
% Some examples based on timeline metaphor. Aggregate methods.
A typical example of timeline visualizations is LifeLines~\cite{Plaisant1996a}, a visualization for personal histories, which uses icons to indicate discrete events and thick horizontal lines for continuous ones. When the number of data items is large, they need to be shown collectively rather than individually. The river metaphor~\cite{Havre2002} is one such method that represents thematic changes in large document collections. Storyline visualizations illustrate the dynamic relationships between entities in a story. The technique was first introduced by Munroe with his hand-drawn visualizations~\cite{Munroe2009}. The visualization summarizes movie plots by depicting each character as a line and each interaction between characters as a converging or diverging bundle of those character lines. Computational layouts have since been introduced to automate the rendering process including work by Tanahashi and Ma~\cite{Tanahashi2012} and Liu et al.~\cite{Liu2013}. 

%There are other metaphors to represent temporal data such as the three-dimensional histogram~\cite{Kosara2004}, the spiral graph~\cite{Weber2001}, or calendar-based approaches~\cite{VanWijk1999}. More detail about timelines or more general time-oriented data visualizations can be found in the comprehensive work of Aigner et al.~\cite{Aigner2011}.

% Skip scalability issue because it's not addressed in SchemaLine
%While the time axis metaphor is simple and intuitive, it can be challenging to visualize a large number of events and the relationships among them along a timeline. In the simplest case, all events are unrelated and visualized independently on the timeline. When several events happened during a short period, multiple layers above the timeline need to created to ensure each event's position matches its time point and no two event descriptions overlap~\cite{Plaisant1998,Pioch2006,Andre2007,Liu2010,Kim2010,Munroe2013,SimileTimeline,TimeGlider}. \note[k]{We mentioned timeline scalability here and need to discuss it for SchemaLine} The resulting timeline can quickly get cluttered as events and layers increase. Several attempts have been made to address the scalability issue including semantics zooming with different levels of detail \cite{Andre2007}, focus and context views \cite{Andre2007, SimileTimeline}, and aggregation of multiple events to one single event \cite{Plaisant1998}.

% Visualize relationships between events
Visualizing individual events on a timeline is relatively simple; however, showing relationships between events is quite challenging. One approach is to explicitly draw an edge between two related entities as in tmViewer~\cite{Kumar1998}. Edge styles can be used to depict different kinds of relationships; however, drawing a node-link diagram on top of the timeline can cause the visualization to become cluttered even with a small number of events. Another method is to use the concurrent perception ability of humans by using color coding or icons to indicate different groupings. When events are distributed along the timeline, this method introduces a heavy cognitive load for viewers to scan through the entire timespan. Our method uses colored backgrounds and clusters all events belonging to the same group to reduce user effort. Relationships within multiple faceted temporal data are addressed by Andr\'{e} et al. in Continuum~\cite{Andre2007} by using views with different scales and the classic details-on-demand technique to save space. More recently, SemaTime~\cite{Stab2010} can visualize two different types of relationship: time-dependent (e.g., lives-in) and time-independent (e.g., father-of). SemaTime stacks events vertically and places related events close together. Time-dependent relationships are depicted by using rectangles crossing the relevant common interval of the two events. Time-independent relationships are illustrated by simple arrows.

% Follow events in timeline
%One essential requirement in timeline visualizations is to be able to follow events chronologically. Even though events are aligned with their time, it can still be difficult to trace which events happen after which, especially when they are close. A conventional rule is that an event is preferentially rendered at the bottom. If the representation of the event (normally a short summary of the event) does not overlap with other event representations, it will stay at the bottom; otherwise, it needs to move up a level. In place of this implicit rule, our method creates an explicit path to guide readers, to improve performance in both time and accuracy. 

%Visual attributes, such as colors and font sizes, are commonly used when events have other attributes (such as grouping and importance level) need to be included in the visualization \cite{TimeGlider}. \note[k]{Similarly, we mentioned number of distinguishable colour issue here and need to discuss it for SchemaLine} However, the number of colors that can be easily distinguished is limited, and so are the levels of possible font sizes to ensure legibility and not taking up too much display estate. 

%To address these problems, LifeLines \cite{Plaisant1998} divides the vertical dimension of timeline into multiple stacks, each to represent a group of events. Continuum \cite{Andre2007} represents the hierarchical relationships between events by displaying child events nested in a parent event. For example, in the music context, a piece belongs to a composer, a composer then lives in an era. All pieces of one composer can be displayed inside a bar of the composer, which represents his/her lifespan.

%Events can have additional complex relationships. In genealogical data, there are relationships such as marriage/divorce and parent-child. \note[k]{can reduce the details about this work if need space}{Kim et al. \cite{Kim2010} represent people as individual lines that cover their lifespans. Vertical axis is used to represent relationships. Unrelated people are displayed vertically far enough to indicate that there is no relationship between them. When two persons get married, the two corresponding lines converge into a bundle to indicate a marriage. Whereas, the diverge of a bundle of two lines denote a divorce. The child is visualized close to the bundle of his parents and a vertical faded dash-line is used to connect the beginning of the child line to his parents.} 

%Another example is movie data, in which a scene contains a start time, a duration and members involved. Munroe creates a hand-drawn visualization to summarize characters' interactions in movies, published in the XKCD webcomic ``Movie Narrative Charts'' \cite{Munroe2013}. Inspired from that work, Tanahashi and Ma~\cite{Tanahashi2012} propose a set of design considerations for aesthetic and legible story-line visualizations, and an algorithm to generate visualizations satisfying these design principles. Similar to representing marriage/divorce relationship in genealogical data,  character lines should go straight unless they converge to or diverge from an interaction. The bundle of lines in an interaction should be adjacent; otherwise, lines must be not adjacent to depict separate character lines. To improve the aesthetics and legibility of the visualization, the algorithm also tries to minimize the number of line wiggles (bends), line crossovers, and white space gaps. The visual representation of SchemaLine is inspired by this work. 

%\subsection{Narrative Construction}
%\label{sec:narrative}
%Segel and Heer \cite{Segel2010} reviewed and classified narrative visualizations based on three dimensions: genre, visual narrative tactics, and narrative structure tactics. They define seven (not mutually exclusive) genres of narrative visualizations including magazine style, annotated chart, partitioned poster, flow chart, comic strip, live show, and film/video/animation. Narrative visualizations can also be classified as author-driven, reader-driven, or hybrid. They give an example of a common approach called "Martini Glass Structure". The story begins with the author's guidance for a while and then leads to the reader-driven stage where they can freely explore the story. Among these, most relevant to our work are those designed for sensemaking and utilize a timeline representation.

\subsection{Timelines within Visual Analytics Systems}
A timeline is commonly integrated into Visual Analytics systems designed for making sense of large and complex datasets including POLESTAR~\cite{Pioch2006}, HARVEST~\cite{Gotz2006}, Jigsaw~\cite{Stasko2007, Gorg2013}, and nSpace2 Sandbox~\cite{Wright2006, SandboxTimeline2012}. 

% visualize what?
To support sensemaking, timelines are typically used to visualize not raw data, but more meaningful information such as user notes (POLESTAR, HARVEST) or extracted entities (Jigsaw, nSpace2 Sandbox) instead. HARVEST visualizes both raw data and synthesized knowledge in one timeline to allow progressive investigation. However, filtering must be supported to prevent valuable information getting lost among dense data.

% manipulation of notes
Most of the systems use timelines to show notes statically, to present a known story instead of dynamically discovering a hidden story. nSpace2 Sandbox is an exception -- it allows users to group related entities into sub-timelines and to alter the entity's date on the timeline if needed. However, one entity cannot be added into multiple timelines, which is necessary when an entity's category is uncertain. Our SchemaLine provides a set of fluid interactions to manipulate notes to build a more semantic schema.

% visual representation of notes and schemata
Notes are typically represented using the ``sticky-notes'' metaphor: a colored rectangle as background with text on top of it. nSpace2 Sandbox provides multiple levels of detail for entities: a short summary, a full article, or even entities of entities. Timelines are commonly visualized as a horizontal axis with notes connecting to the timeline by edges. nSpace2 Sandbox uses a vertical axis timeline as the ``diary'' metaphor with columns for sub-timelines.

% layout
POLESTAR requires manual notes arrangement to fit the display. nSpace2 Sandbox uses a simple linear layout to organize entities, thus entities with nearby dates will overlap on the timeline. Our layout algorithm produces an aesthetically pleasing visualization that avoids this issue while still providing easy note manipulation.

% integration
Timelines are often used as an extra view, coordinated with the whole system. Jigsaw provides a reasoning space called Tablet, where a timeline can be added. nSpace2 Sandbox also introduces a separate component called Timeline view. Even though entities from data space can be dropped into timeline space, it may introduce a heavy cognitive load for users to switch between two working spaces. In the evaluation of this paper, we integrate SchemaLine into an existing system seamlessly to provide concurrent exploration and sensemaking with data.

%is a collaborative knowledge management and sense-making tool for intelligence analysts. Users can extract text and take notes when reading documents. Notes are then placed in an empty space to allow analysts to organize and cluster information. This information can be organized in a timeline; however, the user needs to arrange it manually. Without automatic layout and grouping of related information, it is difficult to assist users in sensemaking.
%
%\note[p]{miss 1,3,4}
%HARVEST~\cite{Gotz2006} allows users to interactively define new knowledge when analyzing data. The synthesized information can then be visualized together with raw data on the timeline. This feature could be useful because insight can be progressively used to gain deeper understanding. However, it may cause the findings get lost among dense data; only selective information should be shown on the timeline. The system does not support linking or grouping synthesized knowledge to produce alternative explanations about the case.
%
%\note[p]{miss 3,4,5}
%Jigsaw~\cite{Stasko2007} is one of the most popular Visual Analytics systems for making sense of a large corpus of documents. It can identify various types of entity (people, place, organizations, etc.) within the corpus and present the complex relationships through multiple linked visualizations. Jigsaw also supports a timeline for sensemaking~\cite{Gorg2013} within a feature called ``tablet''. Tablet is an empty canvas, which allows analysts to freely organize information, create links between entities or take notes. When entities are dropped onto the timeline, they will be automatically organized based on their associated date. A disadvantage of the tablet is that the user needs to open another space to enter notes instead of directly inside the data exploration space. The tablet timeline does not support visually showing different groups of related notes, which is quite important when connecting them together to construct a more cohesive explanation. It is also not clear how this timeline is used to support analysis; it is more often used for reporting the story once discovered~\cite{Liu2010}.
%
%\note[p]{miss nothing!!!} \note{maybe we can say how schemaline is different instead?}
%nSpace Sandbox~\cite{Wright2006} is a commercial sensemaking environment that supports extracting information from different sources, organizing it flexibly and linking these pieces of information. Sandbox's timeline~\cite{SandboxTimeline2012} allows the assembly of different types of evidence, such as documents or pictures, along the time axis. It also supports creating \textit{bands} within the timeline to store different groups of events. Relationships between events within bands or within the timeline are visualized by arrows. Sandbox's timeline is quite advanced in terms of analytical features compared to the timelines in other visual analytics systems. However, the simplistic linear timeline layout is not space-efficient and also makes it difficult to visualize close events. Zooming feature is provided to partially address this issue.

%Visualizing documents along a timeline is one of its many features, but it does not directly support group or interactive editing.
%\note[k]{this needs to be updated for later/latest version of GeoTime. Also the focus should be 'timeline' and not 'narrative'} GeoTime \cite{SandboxTimeline2013} is a Visual Analytics system that is designed for the detection of spatial and temporal pattern within the data. It allows annotated visualizations, which can then be used to build a visual narrative. However, the visual narrative is mainly designed for reporting, and there is limited functionalities for the discovery process of the narratives.
%Aruvi~\cite{Shrinivasan2008} is a Visual Analytics system designed to support sensemaking. It allows the creation of notes about any finding and the visualization leads to it. Such notes can be connected to indicate the relationship between them. However, it only provide free-form organization of the notes and there is no support for timeline visualization. 
%HARVEST \cite{Shrinivasan2009} is a Visual Analytics system that adopts a similar approach to sensemaking as the Aruvi. While it uses an algorithm to find relevant discoveries and views for suggestion, there is no significant changes in the support of sensemaking. As a result, it lacks the ability to automatically layout the finding notes or the support for the discovery of any temporal pattern.
%\note[k]{Phong, can you say something about the timeline in 'Sense.us'?} Sense.us \cite{Heer2009} is one of the early attempts of online collaborative visualization. It allows users to annotate visualizations and use them in the collaborative discussion. The trail of annotated visualizations are then used to construct the trail of a topic. This method is useful to record the evolution of the discussions, but it does not indicate the relationships among the discussion, i.e., how they form a narrative, and can make it difficult to understand a large discussion.
\note[p]{I have no idea with timeline in 'i2 analyst notebook'}
