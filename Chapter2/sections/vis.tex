\chapter{Literature Review}

\graphicspath{{Chapter2/figures/}}

 \section{Sensemaking}
 Sensemaking is described as the process of comprehension, finding meaning and gaining insight from information, producing new knowledge and informing further action. It has been studied in different contexts such as human-computer interaction~\cite{Russell1993}, information science~\cite{Dervin1983}, and organizational studies~\cite{Weick1995}. Pirolli and Card offer a notional model of sensemaking~\cite{Pirolli2005}, describing a cyclic process involving searching and extracting information, representing it in schemata, manipulating the schemas to form and test hypotheses, then presenting the outcome. Klein et al.~\cite{Klein2006} offer a \textit{data--frame} model describing the interaction of data -- the aspects of the world, and frames -- the accounts of the ``sensemaker'' for the situation. The model consists of a few interconnected iterative processes, each for a certain type of sensemaking activity such as fitting data into a frame.

The notion of sensemaking relates to the way in which human process and interpret information about the world, leading to the creation of new knowledge or insight, which informs further action ~\cite{Pirolli2005}. There are a number of different sensemaking theories, which draw attention from scholars in human-computer interaction, psychology and other areas that consider sensemaking in different contexts. Examples include Dervin~\cite{Dervin1983}, who considers how sensemaking is related to information seeking behaviors and needs, and Weick~\cite{Weick1995}, who was concerned with how sensemaking takes place in organizational settings, from individual and social contexts. Pirolli and Card~\cite{Pirolli2005} offer a notional model of sensemaking, describing a cyclic process involving representations of information in \textit{schemata} and the manipulation of \textit{schemas} to gain insight forming some knowledge or understanding. Klein et al.~\cite{Klein2006} offer the \textit{Data/Frame} theory describing the interaction of \textit{data}, which are aspects of the world and \textit{frames}, that are the sensemaker’s representations of the situation. 

\section{Visualization}
\section{Analytic Provenance}

%\subsection{Timeline Visualization}
% Timeline visualization
%Timeline is one of the earliest visualizations. Back in 1765, Joseph Priestley created the Chart of Biography about two thousand famous people from 1200 B.C to 1750 A.D~\cite{Priestley1765} -- one of the oldest documented timeline. Along a horizontal timeline, Priestley used bars to depict lifespans and dots to illustrate the uncertain birth and dead dates. This remains one of the most popular visual metaphors for timeline visualization: time is represented as a horizontal axis; an event is plotted as a point if it is an instant or a bar spans its interval otherwise. Geographical information can also be included in timeline visualizations as in the classic information visualization example of Napoleon's March on Moscow in 1812-1813 by Charles Joseph Minard~\cite{Minard1869}. 

In this section, we consider related work on representing temporal data and producing timelines, before also examining larger visual analytic systems that include timelines in their feature set.

\subsection{Timeline Visualizations}
% Some examples based on timeline metaphor. Aggregate methods.
A typical example of timeline visualizations is LifeLines~\cite{Plaisant1996a}, a visualization for personal histories, which uses icons to indicate discrete events and thick horizontal lines for continuous ones. When the number of data items is large, they need to be shown collectively rather than individually. The river metaphor~\cite{Havre2002} is one such method that represents thematic changes in large document collections. Storyline visualizations illustrate the dynamic relationships between entities in a story. The technique was first introduced by Munroe with his hand-drawn visualizations~\cite{Munroe2009}. The visualization summarizes movie plots by depicting each character as a line and each interaction between characters as a converging or diverging bundle of those character lines. Computational layouts have since been introduced to automate the rendering process including work by Tanahashi and Ma~\cite{Tanahashi2012} and Liu et al.~\cite{Liu2013}. 

%There are other metaphors to represent temporal data such as the three-dimensional histogram~\cite{Kosara2004}, the spiral graph~\cite{Weber2001}, or calendar-based approaches~\cite{VanWijk1999}. More detail about timelines or more general time-oriented data visualizations can be found in the comprehensive work of Aigner et al.~\cite{Aigner2011}.

% Skip scalability issue because it's not addressed in SchemaLine
%While the time axis metaphor is simple and intuitive, it can be challenging to visualize a large number of events and the relationships among them along a timeline. In the simplest case, all events are unrelated and visualized independently on the timeline. When several events happened during a short period, multiple layers above the timeline need to created to ensure each event's position matches its time point and no two event descriptions overlap~\cite{Plaisant1998,Pioch2006,Andre2007,Liu2010,Kim2010,Munroe2013,SimileTimeline,TimeGlider}. \note[k]{We mentioned timeline scalability here and need to discuss it for SchemaLine} The resulting timeline can quickly get cluttered as events and layers increase. Several attempts have been made to address the scalability issue including semantics zooming with different levels of detail \cite{Andre2007}, focus and context views \cite{Andre2007, SimileTimeline}, and aggregation of multiple events to one single event \cite{Plaisant1998}.

% Visualize relationships between events
Visualizing individual events on a timeline is relatively simple; however, showing relationships between events is quite challenging. One approach is to explicitly draw an edge between two related entities as in tmViewer~\cite{Kumar1998}. Edge styles can be used to depict different kinds of relationships; however, drawing a node-link diagram on top of the timeline can cause the visualization to become cluttered even with a small number of events. Another method is to use the concurrent perception ability of humans by using color coding or icons to indicate different groupings. When events are distributed along the timeline, this method introduces a heavy cognitive load for viewers to scan through the entire timespan. Our method uses colored backgrounds and clusters all events belonging to the same group to reduce user effort. Relationships within multiple faceted temporal data are addressed by Andr\'{e} et al. in Continuum~\cite{Andre2007} by using views with different scales and the classic details-on-demand technique to save space. More recently, SemaTime~\cite{Stab2010} can visualize two different types of relationship: time-dependent (e.g., lives-in) and time-independent (e.g., father-of). SemaTime stacks events vertically and places related events close together. Time-dependent relationships are depicted by using rectangles crossing the relevant common interval of the two events. Time-independent relationships are illustrated by simple arrows.

% Follow events in timeline
%One essential requirement in timeline visualizations is to be able to follow events chronologically. Even though events are aligned with their time, it can still be difficult to trace which events happen after which, especially when they are close. A conventional rule is that an event is preferentially rendered at the bottom. If the representation of the event (normally a short summary of the event) does not overlap with other event representations, it will stay at the bottom; otherwise, it needs to move up a level. In place of this implicit rule, our method creates an explicit path to guide readers, to improve performance in both time and accuracy. 

%Visual attributes, such as colors and font sizes, are commonly used when events have other attributes (such as grouping and importance level) need to be included in the visualization \cite{TimeGlider}. \note[k]{Similarly, we mentioned number of distinguishable colour issue here and need to discuss it for SchemaLine} However, the number of colors that can be easily distinguished is limited, and so are the levels of possible font sizes to ensure legibility and not taking up too much display estate. 

%To address these problems, LifeLines \cite{Plaisant1998} divides the vertical dimension of timeline into multiple stacks, each to represent a group of events. Continuum \cite{Andre2007} represents the hierarchical relationships between events by displaying child events nested in a parent event. For example, in the music context, a piece belongs to a composer, a composer then lives in an era. All pieces of one composer can be displayed inside a bar of the composer, which represents his/her lifespan.

%Events can have additional complex relationships. In genealogical data, there are relationships such as marriage/divorce and parent-child. \note[k]{can reduce the details about this work if need space}{Kim et al. \cite{Kim2010} represent people as individual lines that cover their lifespans. Vertical axis is used to represent relationships. Unrelated people are displayed vertically far enough to indicate that there is no relationship between them. When two persons get married, the two corresponding lines converge into a bundle to indicate a marriage. Whereas, the diverge of a bundle of two lines denote a divorce. The child is visualized close to the bundle of his parents and a vertical faded dash-line is used to connect the beginning of the child line to his parents.} 

%Another example is movie data, in which a scene contains a start time, a duration and members involved. Munroe creates a hand-drawn visualization to summarize characters' interactions in movies, published in the XKCD webcomic ``Movie Narrative Charts'' \cite{Munroe2013}. Inspired from that work, Tanahashi and Ma~\cite{Tanahashi2012} propose a set of design considerations for aesthetic and legible story-line visualizations, and an algorithm to generate visualizations satisfying these design principles. Similar to representing marriage/divorce relationship in genealogical data,  character lines should go straight unless they converge to or diverge from an interaction. The bundle of lines in an interaction should be adjacent; otherwise, lines must be not adjacent to depict separate character lines. To improve the aesthetics and legibility of the visualization, the algorithm also tries to minimize the number of line wiggles (bends), line crossovers, and white space gaps. The visual representation of SchemaLine is inspired by this work. 

%\subsection{Narrative Construction}
%\label{sec:narrative}
%Segel and Heer \cite{Segel2010} reviewed and classified narrative visualizations based on three dimensions: genre, visual narrative tactics, and narrative structure tactics. They define seven (not mutually exclusive) genres of narrative visualizations including magazine style, annotated chart, partitioned poster, flow chart, comic strip, live show, and film/video/animation. Narrative visualizations can also be classified as author-driven, reader-driven, or hybrid. They give an example of a common approach called "Martini Glass Structure". The story begins with the author's guidance for a while and then leads to the reader-driven stage where they can freely explore the story. Among these, most relevant to our work are those designed for sensemaking and utilize a timeline representation.

\subsection{Timelines within Visual Analytics Systems}
A timeline is commonly integrated into Visual Analytics systems designed for making sense of large and complex datasets including POLESTAR~\cite{Pioch2006}, HARVEST~\cite{Gotz2006}, Jigsaw~\cite{Stasko2007, Gorg2013}, and nSpace2 Sandbox~\cite{Wright2006, SandboxTimeline2012}. 

% visualize what?
To support sensemaking, timelines are typically used to visualize not raw data, but more meaningful information such as user notes (POLESTAR, HARVEST) or extracted entities (Jigsaw, nSpace2 Sandbox) instead. HARVEST visualizes both raw data and synthesized knowledge in one timeline to allow progressive investigation. However, filtering must be supported to prevent valuable information getting lost among dense data.

% manipulation of notes
Most of the systems use timelines to show notes statically, to present a known story instead of dynamically discovering a hidden story. nSpace2 Sandbox is an exception -- it allows users to group related entities into sub-timelines and to alter the entity's date on the timeline if needed. However, one entity cannot be added into multiple timelines, which is necessary when an entity's category is uncertain. Our SchemaLine provides a set of fluid interactions to manipulate notes to build a more semantic schema.

% visual representation of notes and schemata
Notes are typically represented using the ``sticky-notes'' metaphor: a colored rectangle as background with text on top of it. nSpace2 Sandbox provides multiple levels of detail for entities: a short summary, a full article, or even entities of entities. Timelines are commonly visualized as a horizontal axis with notes connecting to the timeline by edges. nSpace2 Sandbox uses a vertical axis timeline as the ``diary'' metaphor with columns for sub-timelines.

% layout
POLESTAR requires manual notes arrangement to fit the display. nSpace2 Sandbox uses a simple linear layout to organize entities, thus entities with nearby dates will overlap on the timeline. Our layout algorithm produces an aesthetically pleasing visualization that avoids this issue while still providing easy note manipulation.

% integration
Timelines are often used as an extra view, coordinated with the whole system. Jigsaw provides a reasoning space called Tablet, where a timeline can be added. nSpace2 Sandbox also introduces a separate component called Timeline view. Even though entities from data space can be dropped into timeline space, it may introduce a heavy cognitive load for users to switch between two working spaces. In the evaluation of this paper, we integrate SchemaLine into an existing system seamlessly to provide concurrent exploration and sensemaking with data.

%is a collaborative knowledge management and sense-making tool for intelligence analysts. Users can extract text and take notes when reading documents. Notes are then placed in an empty space to allow analysts to organize and cluster information. This information can be organized in a timeline; however, the user needs to arrange it manually. Without automatic layout and grouping of related information, it is difficult to assist users in sensemaking.
%
%\note[p]{miss 1,3,4}
%HARVEST~\cite{Gotz2006} allows users to interactively define new knowledge when analyzing data. The synthesized information can then be visualized together with raw data on the timeline. This feature could be useful because insight can be progressively used to gain deeper understanding. However, it may cause the findings get lost among dense data; only selective information should be shown on the timeline. The system does not support linking or grouping synthesized knowledge to produce alternative explanations about the case.
%
%\note[p]{miss 3,4,5}
%Jigsaw~\cite{Stasko2007} is one of the most popular Visual Analytics systems for making sense of a large corpus of documents. It can identify various types of entity (people, place, organizations, etc.) within the corpus and present the complex relationships through multiple linked visualizations. Jigsaw also supports a timeline for sensemaking~\cite{Gorg2013} within a feature called ``tablet''. Tablet is an empty canvas, which allows analysts to freely organize information, create links between entities or take notes. When entities are dropped onto the timeline, they will be automatically organized based on their associated date. A disadvantage of the tablet is that the user needs to open another space to enter notes instead of directly inside the data exploration space. The tablet timeline does not support visually showing different groups of related notes, which is quite important when connecting them together to construct a more cohesive explanation. It is also not clear how this timeline is used to support analysis; it is more often used for reporting the story once discovered~\cite{Liu2010}.
%
%\note[p]{miss nothing!!!} \note{maybe we can say how schemaline is different instead?}
%nSpace Sandbox~\cite{Wright2006} is a commercial sensemaking environment that supports extracting information from different sources, organizing it flexibly and linking these pieces of information. Sandbox's timeline~\cite{SandboxTimeline2012} allows the assembly of different types of evidence, such as documents or pictures, along the time axis. It also supports creating \textit{bands} within the timeline to store different groups of events. Relationships between events within bands or within the timeline are visualized by arrows. Sandbox's timeline is quite advanced in terms of analytical features compared to the timelines in other visual analytics systems. However, the simplistic linear timeline layout is not space-efficient and also makes it difficult to visualize close events. Zooming feature is provided to partially address this issue.

%Visualizing documents along a timeline is one of its many features, but it does not directly support group or interactive editing.
%\note[k]{this needs to be updated for later/latest version of GeoTime. Also the focus should be 'timeline' and not 'narrative'} GeoTime \cite{SandboxTimeline2013} is a Visual Analytics system that is designed for the detection of spatial and temporal pattern within the data. It allows annotated visualizations, which can then be used to build a visual narrative. However, the visual narrative is mainly designed for reporting, and there is limited functionalities for the discovery process of the narratives.
%Aruvi~\cite{Shrinivasan2008} is a Visual Analytics system designed to support sensemaking. It allows the creation of notes about any finding and the visualization leads to it. Such notes can be connected to indicate the relationship between them. However, it only provide free-form organization of the notes and there is no support for timeline visualization. 
%HARVEST \cite{Shrinivasan2009} is a Visual Analytics system that adopts a similar approach to sensemaking as the Aruvi. While it uses an algorithm to find relevant discoveries and views for suggestion, there is no significant changes in the support of sensemaking. As a result, it lacks the ability to automatically layout the finding notes or the support for the discovery of any temporal pattern.
%\note[k]{Phong, can you say something about the timeline in 'Sense.us'?} Sense.us \cite{Heer2009} is one of the early attempts of online collaborative visualization. It allows users to annotate visualizations and use them in the collaborative discussion. The trail of annotated visualizations are then used to construct the trail of a topic. This method is useful to record the evolution of the discussions, but it does not indicate the relationships among the discussion, i.e., how they form a narrative, and can make it difficult to understand a large discussion.
\note[p]{I have no idea with timeline in 'i2 analyst notebook'}






\section{Related Work}

\subsection{Timeline Visualizations}
The most common form of timeline visualization uses a horizontal axis to represent time progressing from left to right, with events positioned horizontally according to their timestamps. A well known example is LifeLines~\cite{Plaisant1998} -- a visualization of personal medical records. LifeLines uses icons to indicate discrete events and thick horizontal lines for continuous ones. Timelines can be integrated into a tree format to represent changes in a hierarchy over time as in TimeTree~\cite{Card2006}. Geographical information can also be embedded in timelines as in the classic visualization of Napoleon's March in Moscow in 1812--1813 by Charles Joseph Minard~\cite{Minard1869}. The book by Aigner et al.~\cite{Aigner2011} provides a comprehensive review of timelines and other time-oriented data visualizations.

Techniques such as aggregation and interaction are commonly used when there are a large number of events. LifeLines~\cite{Plaisant1998} aggregates events to save display estate; for example, a series of similar prescriptions can be grouped together. ThemeRiver~\cite{Havre2002} or Streamgraph~\cite{Byron2008} uses a river metaphor to represent aggregated changes of themes over time in a large document collection. Each river is a theme, and its width at certain time points shows the number of documents in that theme. Common interaction techniques are often used in the visualization of large timelines to support their exploration, including overview+detail \cite{Andre2007},  filtering~\cite{Plaisant1996a}, and details-on-demand~\cite{Stab2010}.

\subsection{Set Relations in Timelines}
According to the Gestalt principles of grouping, humans naturally perceive objects as a whole rather than as the sum of their parts~\cite{Koffka1935}. Three of the principles are commonly used to show set relationships among events: similarity, proximity, and uniform connectedness.

The principle of \textit{similarity} states that objects are perceptually grouped together if they are similar to each other~\cite{Koffka1935}. This principle is extensively applied to show set relations in timelines by using colors and shapes. Time indicators as icons (time-point events)~\cite{SimileTimeline2009} and bars (interval events)~\cite{Wang2008} are colored according to event set memberships. Different shapes for icons~\cite{TimeGlider2012} and bars~\cite{Plaisant1998} are also used to distinguish set memberships. It is more challenging to represent multiple set memberships. LineSets~\cite{Alper2011} uses concentric circles for icons, where each circle is colored to represent one set.

According to the \textit{proximity} principle, objects that are close together are perceived to be more related than objects that are spaced further apart~\cite{Koffka1935}. In Chart of Biography~\cite{Priestley1765}, people within a category are placed in a horizontal band, away from people in other categories. LifeLines~\cite{Plaisant1998} splits medical records into different sets, such as \textit{medication} or \textit{diagnosis}, and places them into vertically stacks, which works well if no two sets overlap. Storyline visualizations~\cite{Tanahashi2012,Liu2013} use curved lines to show interactions among characters within the movie timeline. Character lines converge to a bundle if they appear in the same interaction, and diverge when the it ends. Each line can be considered as a set passing through all of its members, and each interaction is a multi-set event. Thus, this method only works for interval events.

Elements tend to be grouped together if they are visually connected~\cite{Palmer1994}. Following this \textit{uniform connectedness} principle, SchemaLine~\cite{Nguyen2014} draws a rectilinear path connecting events belonging to a same set together. Also, tmViewer~\cite{Kumar1998} links related entities with line segments. Different line colors, thicknesses, and styles were used to distinguish set relations. This method can show events with multiple set memberships by connecting them with multiple edges. However, extra edges and crossings may negatively impact the readability of the timeline.

When similarity and proximity are applied together, the later principle dominates~\cite{Ware2013}. Moreover, uniform connectedness is stronger than proximity~\cite{Palmer1994}. For example, objects with different colors and shapes but located close together are more likely to be perceived as a group, and distant objects but with a closed contour surrounding them also provide a strong sense of grouping. Applying these ideas to visualize set relations for timelines, methods relying on similarity such as colored icons~\cite{Wang2008} are less effective than spatial grouping methods such as LifeLines~\cite{Plaisant1996a}. And those, in turn, are less effective than methods using line segments such as tmViewer~\cite{Kumar1998}. 

\subsection{Set Visualizations}
Sets and their relationships can be visualized using Venn~\cite{Ruskey1997} or Euler~\cite{Rodgers2014} diagrams. Simonetto et al.~\cite{Simonetto2009} proposed a technique to automatically visualize sets that were previously not possible with Euler diagrams. However, the complex shapes it produces may reduce visualization readability. In their controlled study, Riche and Dwyer~\cite{Riche2010} showed that for complex set intersections, duplications of shared elements resulted in a better performance in readability tasks than a none-duplicated visualization with more complex shapes. 

These methods assume the positions of set elements are not fixed, which reduces their applicability for geo-located or timeline events. Techniques without such constraints include Bubble Sets~\cite{Collins2009a}, LineSets~\cite{Alper2011}, and KelpFusion~\cite{Meulemans2013}. These methods employ the connectedness principle of the Gestalt laws~\cite{Palmer1994} by connecting set elements using extra visual elements. Bubble Sets draws an iso-contour surrounding elements within a set. This iso-contour is filled with a semi-transparent color so that the intersection between sets is shown as an area of blended color. Collins et al.~\cite{Collins2009a} provided an example of applying Bubble Sets to a timeline, in which case a force-directed algorithm is used to adjust the vertical positions of elements while the horizontal position along the time axis is fixed. 

LineSets applies a B\'{e}zier curve to connect data items. The curve follows the shortest path passing through all elements in the set. Its study showed that LineSets outperforms Bubble Sets in certain readability tasks ~\cite{Alper2011}. KelpFusion, a hybrid technique, uses lines for data-sparse areas and surfaces for data-dense areas. The results of an evaluation on readability tasks~\cite{Meulemans2013} demonstrated that it outperforms Bubble Sets in both accuracy and completion time, and outperforms LineSets in completion time. There has been no reported attempt to apply LineSets or KelpFusion to timeline visualizations. It is expected that crossings between lines, areas and the event text may reduce the timeline readability.







\section{Related Work}

\subsection{Sensemaking and Qualitative Analysis}
Qualitative research methodologies~\cite{adams2008qualititative} are commonly used in study of sensemaking. They allow researchers to reveal often complex user experiences and understand issues that are experienced subjectively or collectively~\cite{Pace2004327, adams2008qualititative}. Moreover, sensemaking research is often concerned not with testing an existing theory, but building a new one through the collection and analysis of relevant data, generating new knowledge about users and the usage of technology~\cite{rogers2012hci}.

There are a number of inductive approaches to qualitative research popular in sensemaking studies, such as \textit{grounded theory}~\cite{corbin1994grounded}, \textit{content analysis}~\cite{stemler2001overview}, and \textit{thematic analysis}~\cite{guest2011applied} that rely on the interpretation of rich textual and multimedia data. There is commonality in these approaches in that they all require a level of manual processing of data, coding and indexing it before describing it in the context of categories or themes. Furthermore, in the case of multimedia data, transcription of audio or video data is often also required. Though these approaches lead to important insights, they are labor intensive, time-consuming and costly in their application \cite{wong2002analysing}. There are a number of widely used software packages which are designed for researches using an inductive approach, offering ways to code and index data in various formats \cite{lewins2007using}. These tools give qualitative researchers useful ways of managing data, however, a qualitative analysis is still a largely manual process which requires a substantial investment of time and resources in leading to insightful findings.

\subsection{Analytic Provenance for Sensemaking}
Analytic provenance describes the interactive data exploration using a visual analytics system and the human sensemaking during that process. Besides the four semantic layers~\cite{Gotz2009} discussed earlier, it also includes the \textit{7W information}~\cite{provenance-7w} (Who, What, Where, Why, When,
Which, and hoW), which were initially proposed for \emph{data provenance}~\cite{data-provenance-survey-sigmod,data-provenance-survey-acm, data-provenance-survey-2008} that focuses on the data collection and computational analysis. This provides the context necessary for interpretation, such as authors, creation time, input data, and visualization parameters. Similar to other scientific workflows, an analytic provenance pipeline consists of capture, storage, analysis and visualization. Heer et al.~\cite{Heer2008} discussed design considerations for such pipeline, which covers underlying provenance's models, visual representations, and operations for analysis. The following discussions focus on the capture and visualization, in the context of recovering of user's sensemaking process from analytic provenance.

\subsubsection{Capture}
There is limited literature on capturing \emph{events} because it is relatively easy and provides little semantics alone. Among these, Glass Box~\cite{Cowley2006} can record a wide range of low-level \textit{events} such as mouse clicks, key strokes, and window events. Its objective is to capture and archive intelligence analysis activities so they can be retrieved later. On a higher semantic level, \textit{Actions}, such as changing the visualization settings and sorting a dataset, are usually performed through interactive widgets including menus and buttons. In theory, it is straightforward to capture them if a visualization system intends to do so. Some systems~\cite{Shrinivasan2008} maintain an action history to support undo and redo, which is an example of utilizing the actions' provenance.

Capturing \textit{task} and \textit{sub-task} is usually more challenging. As previously mentioned, such information is usually part of users' thinking that a visual analytics system does not have direct access to. Existing methods either take a manual or automatic approach. Manual approaches encourage users to record their thinking during sensemaking. However, this will not work if the method introduces considerable distraction or does not offer any benefits. Allowing user annotation is one of the most common forms~\cite{diva,schemaline}: the user creates \emph{notes} or \emph{annotations} that record comments, findings, or hypotheses. Those notes can be associated with the visualization, allowing users returning to the states when the notes were made~\cite{Pike2007, Shrinivasan2008} to re-examine the context or investigate further. The incentive to users is that such annotation digitizes a common sensemaking activity (i.e., note taking) and allows for features such as searching.  This also integrates notes with their visualisation context, allow for better interpretation. However, annotations are unlikely to cover all the analytical thinking. For example, users are more likely to record the findings they made than the process or approach that led them there. To encourage user to write richer notes, a visual analytic system needs to provide additional benefits such as the ability to create visual narratives~\cite{diva} that reveals the reasoning process and help users review and plan exploratory analysis for complex sensemaking task after recording the current progress~\cite{Lunzer2014}.

One of the main disadvantages of manual capture is the requirement of direct input from users. Automatic approaches try to address this by inferring higher level analytic provenance from what can be automatically captured including event and action provenance. However, this turns out to be a difficult task. An experiment studied the effectiveness of manual recovering of reasoning process from a user's action provenance, and the results showed that about 60\% to 70\% of high level analytic provenance can be correctly recovered~\cite{Dou2009}. Given the difficulty, a few methods attempted to partially uncover the high level analytic provenance. One such example is \textit{action chunking}, i.e., identify a group of actions that are likely to be part of the same sub-task, without knowing what the sub-task is~\cite{Gotz2009}. Such approaches apply heuristics to infer patterns from action logs based on repeated occurrence and proximity in data/visualization space or analysis time. More recently, there has been advancement in developing an automated and real-time technique to learn about users~\cite{Brown2014}. Based on very low-level events, mouse clicks and movements, collected from a visual search task, the algorithms can detect whether a user would be fast or slow at completing the task with 62\% to 83\% accuracy. They can also infer some user traits including locus of control, extraversion and neuroticism with 61\% to 67\% accuracy.

Deriving user thinking from their interaction data can be of value beyond understanding sensemaking and is common in other domains. For example, many websites record user browsing interactions in hope to derive higher level information such as user goals and needs. Data mining approaches are commonly used to detect patterns within such data~\cite{Cooley1997}, which can then be used to provide better service such as recommendations~\cite{Wei2007}. Chi et al.~\cite{Chi2001} proposed an algorithm to infer user needs from user's surfing patterns based on the concept of \textit{information scent}, which is the perception of the value and cost of information sources obtained from proximal cues with respect to the goal of the user~\cite{Pirolli1999}.

\subsubsection{Visualization}
Analytic provenance visualization is commonly used to provide an overview of the sensemaking process or reveal any patterns during this process, both of which can help researchers to understand users' thinking. Node-link diagrams are a popular choice to show an overview of the sensemaking process~\cite{vistrails,Pike2007,Shrinivasan2008,Kadivar2009,harvest,Dunne2012}. In most cases, nodes represent system states and edges are actions that transition the system from one state to another. Once an analytic provenance network is created, graph layout algorithms can be applied to improve the visualization. A sensemaking session can have hundreds of system states, and the analytic provenance network usually needs to share the display estate with other visualizations. As a result, it can be challenging to show the entire network within a limited space. It is possible to apply techniques for visualizing large networks such as clustering and aggregation. However, this needs to be done in a way that does not lose the information important for understanding the sensemaking process. To the best of our knowledge, we are not aware of any work addressing this challenge yet.

Besides the overall sensemaking process, the details of each system state and user actions are important for recovery of users' thinking. To provide more context, the common approach is \emph{detail on demand}: when a sensemaking step is selected, the visual analytics system shows the corresponding visualization state and the action's information ~\cite{Pike2007,Shrinivasan2008,Kadivar2009}. This works well if a visual analytics system allows user to go back to a previous state: showing the sensemaking context essentially restores all the visualization views to a previous state. However, to understand a sequence of sensemaking actions, researchers must go through one step at a time, sometimes back and forth. This can introduce extra cognitive work load on the user's memory, thus slow down the analysis. Methods such as GraphTrail~\cite{Dunne2012} show the full details of multiple system states and the links between them at the same time. By zooming and panning, users can choose to see more sensemaking steps (with less detail) or more information about individual system state (but less states simultaneously). This method works well when the visualization state is simple, e.g., only has one view. When a system consists of multiple visualizations, it becomes difficult to see the details of each state when more than one states are shown.

Chronicle~\cite{Grossman2010} utilizes a similar interface to SensePath, but with different design intentions. It captures the entire editing history of a graphical document to allow further study of how a complex image product is accomplished. In contrast, the goal of SensePath is to help analysts recover the high level sensemaking process. Following a similar approach, Delta~\cite{Kong2012} allows comparing different editing workflows to choose the most suitable one by visualizing the steps performed in those workflows. 



\section{Related Work}
Analytic provenance describes the interactive data exploration with a visual analytics system and the human sensemaking during that process~\cite{Xu2015}. Similar to scientific workflows, an analytic provenance pipeline consists of capture, storage, visualization and analysis~\cite{North2011}. Heer et al.~\cite{Heer2008} discuss design considerations for such a pipeline. The following discussions focus on the capture, visualization and utilization in the context of supporting sensemaking.

\subsection{Capture}
Gotz and Zhou~\cite{Gotz2009} divide analytic provenance into four layers according to its semantic richness (in descending order): task, sub-task, action and event. Capturing low level events is relatively straightforward but provides little semantics~\cite{Cowley2006}. Capturing provenance at ``action'' level is more common because it can be done automatically but still provides meaningful information~\cite{Shrinivasan2008, Nguyen2016}. However, capturing ``sub-task'' and ``task'' is more challenging because such information is usually part of users' thinking that systems do not have direct access to~\cite{Gotz2009,Xu2014}.

Brehmer and Munzner~\cite{Brehmer2013} provide a multi-level typology of abstract visualization tasks describing why the task is performed, how the task is
performed, and what are the task’s inputs and outputs. This typology breaks the intention of the user when performing actions into a three-level hierarchy, thus makes it more feasible to capture compared to the high level task and domain-dependent ``task'' and ``sub-task'' in the Gotz and Zhou's model.

% Web: What to collect which can support sensemaking
Information discovered during a browser-based sensemaking task can be collected at different levels of granularity: a web page URL~\cite{Baldonado1997,Gotz2007}, a page element such as \textit{table} and \textit{form} HTML tags~\cite{Schraefel2002,Hong2008}, or a specific fragment of text~\cite{Pioch2006,Nguyen2016}. Finer-grained capture allows users to record what they want with higher precision. Besides this manual capture, visited web pages can be recorded automatically as in browser's history feature. Page linking relationships among pages including opening from a web link and using browser's back button can also be captured~\cite{Ayers1995,Hightower1998,Milic-Frayling2003}.

% Web: Smart way to help collect more
% can remove to save space
%Several methods have developed to help users collect relevant information faster. SenseMaker~\cite{Baldonado1997} allows users to retrieve query results based on similarity with given examples such as web pages having the same domain or articles from co-authors. Dontcheva~et~al~\cite{Dontcheva2006} propose a method to extract similar information by first allowing users to collect a web element, then automatically extracting the information pieces having the same web structure as the collected sample. This method only works with web pages sharing the same template. \note[k]{I don't see any direct connection between sensemaker and Dontcheva's work and sensemap. They appear to be purely algorithmic.} ScratchPad~\cite{Gotz2007} computes and visualizes the relevance between the visiting web page and all the captured information, which may help the user determine the effort needs to spend on the web page.

% Cost structure paper: describe the study here
Besides what to capture, when to capture is also an important decision that needs to be made. Kittur et al~\cite{Kittur2013} conducted a user study to explore when is the best time to ask users to structure the captured information: when all the capture is done or at the same time with the capture. The results show that there are no significant difference between two options in terms of total time spent, cognitive workload, or preference. However, curation at a later stage has significantly better structured data because it has fewer dimensions and these dimensions have more commonalities across participants. 

%Kittur et al~\cite{Kittur2013} conducted a user study to characterize the costs and benefits of curation in the sensemaking process. More specifically, they compared two sensemaking strategies: curation completely after collection or curation at the same time with collection. \annote[k]{Information is collected by highlighting text and curated by assigning a dimension (i.e., tagging) and valence (good/neutral/bad). For instance, ``The Canon EOS Rebel T2i is \textit{good} in terms of \textit{lens}''}{this detail is not necessary}. The results show that there are no significant difference between two options in terms of total time spent, cognitive workload, or preference. However, curation later has significantly better structured data in terms of \annote[k]{fewer dimensions, fewer singleton dimensions, and more commonalities of dimensions across participants}{can you explain this without having to mention the details? It is not obvisou why 'fewer singleton dimensions' is better (or even what it is)}. \annote[k]{We wonder how a hybrid approach, let users curate whenever they want, would perform. We think that curating when collection is done may be adversely affected by the limit of users' working memory, especially with hourly length of sensemaking; and curating immediately after collecting a single piece of information loses the overall picture.}{move this to design discussion.}

\subsection{Visualization}
% Visualization
%In the simplest form, the collected information can be displayed as a list of items, each has a text showing the web page title~\cite{Schraefel2002}, or a table with rows for items and columns for attributes such as page title, URL and abstract~\cite{Baldonado1997}. Hunter--Gatherer~\cite{Schraefel2002} also provides a detailed view showing all collected information pieces together, each consists of a web page title, URL, and the actual captured web element such as text, form, image, and video.

%Tree is another option to organize collected information. SSGIS~\cite{Qu2003} can suggest an initial structure for the search results by clustering them into a tree. POLESTAR~\cite{Pioch2006} allows users to organize the collected snippets into a hierarchy and displays them as in ``Window Explorer''. Dontcheva~et~al~\cite{Dontcheva2006} present ``anchor-based'' layouts to display collected information pieces based on their common anchor attribute. For example, a \textit{calendar} layout can use to organize pieces based on their temporal information; and a \textit{map} layout is suitable to show their spatial information. Timeline is also a common metaphor to display temporal information~\cite{Pioch2006, Nguyen2015}.

Analytic provenance visualization is commonly used to provide an overview of the sensemaking process and to reveal any patterns during this process. Tree visualization is typically used for this purpose~\cite{Pike2007,Kadivar2009,Gotz2010,Dunne2012}. In most cases, a vertex represents a system state and an edge is an action transitioning one state to another. A branch indicates that the user revisits a state and performs another action. A long sensemaking session can produce hundreds of system states, thus makes it more challenging to show the entire process within a limited space. Large network visualization techniques such as clustering and aggregation can be applied to address this issue; however, this should not remove the important information for understanding the sensemaking process. Temporal information can be encoded either by color coded vertex~\cite{Bavoil2005} or edge length~\cite{Shrinivasan2008}. A timeline is also a common metaphor to display temporal information~\cite{Pioch2006, Nguyen2015}.

After understanding the overall sensemaking process, it is common to drill down to examine system states more in-depth. The common approach is \emph{detail on demand}: when a sensemaking step is selected, the visual analytics system allows the user to revisit that state~\cite{Pike2007,Shrinivasan2008,Kadivar2009}. Understanding a sequence of sensemaking actions is more difficult because it requires stepping through all of the actions, sometimes back and forth. This can introduce extra cognitive work load on the user's memory, thus slowing down the analysis. To address this issue, all states can be displayed at once as in Image Graph~\cite{Ma1999} and GraphTrail~\cite{Dunne2012}. However, it worsens the scalability issue.

\subsection{Utilization}
Analytic provenance supports visual narrative construction, during which the user composes findings into a cohesive story. A narrative can include provenance information at different levels: an analysis result, user notes, visualizations and raw data. DIVA~\cite{Walker2013} allows users to create a narrative based on user annotations and captured visualization states, and makes it possible to revisit the visualizations as when they were captured. SchemaLine~\cite{Nguyen2014} enables narrative construction grouping user notes along the timeline.

It is also possible to understand the sensemaking processes of other people by analyzing their analytic provenance~\cite{Dou2010}. SensePath~\cite{Nguyen2016} captures user sensemaking actions and provides a set of visualization and analysis tools allowing other people such as HCI researchers to explore the user sensemaking process through analysis of the captured actions.

% Tagging, node movement, links
%Captured information can be assigned labels or tags to attach meaning and then can be browsed and searched later~\cite{Hong2008, Ryder2010}. This could be a useful starting step in the curation process, which provides the user a semantic structure of the collected information. Other approaches allowing users to create relations among the collected pieces is to freely organize them in the display space~\cite{Pioch2006}  and to draw links connecting related pieces~\cite{Gotz2007}.

% Reasoning
To support further analysis, visual analytics systems commonly provide a ``reasoning workspace'', where the captured information can be freely spatially organized and connected by adding links~\cite{Shrinivasan2008, Xu2014}. Formal analytic methods for reasoning can also be supported. POLESTAR~\cite{Pioch2006} uses a graphical approach to support Toulmin argumentation~\cite{Toulmin2003}: it represents arguments as a tree structure of supporting/rebutting claims, each powered by at least one piece of evidence. Sandbox~\cite{Wright2006} supports analysis of competing hypotheses by assigning each supporting/counter evidence of all the hypotheses a score based on its relevance  and computing the final score to support user decision making.

% Presentation/Organization/Story-telling
After solving a sensemaking task, users may need to present their findings; for instance, to their colleagues to share the knowledge or to their managers as part of the report. They may also need to customize their story depending on different audience's needs and backgrounds. Sandbox~\cite{Wright2006} generates a report by simply exporting curated collections to HTML. Diigo~\footnote{\url{https://www.diigo.com/}}, an online bookmarking tool, allows the user to combine the collected information with their own notes to produce a more organized document with supporting information.

% Share: too trivial?
%Another common way to communicate the findings is through sharing. A (curated) collection can be exported as a local file~\cite{Dontcheva2006} or an online URL~\cite{Schraefel2002}. Social bookmarking also allows users to collect information but considers sharing to community in the first place.\note[k]{I dont' understand} Delicious~\footnote{\url{http://delicious.com/}} is one of the earliest bookmarking services allowing users to save and share web pages of interest and categorize them by tagging. It also attracts attention from academics such as discussing design principles for a social bookmarking service in a large enterprise addressing online identity, privacy and information discovery issues~\cite{Millen2006}. \note[k]{why is this relevant?}
