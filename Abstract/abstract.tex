% ************************** Thesis Abstract *****************************
% Use `abstract' as an option in the document class to print only the titlepage and the abstract.
\begin{abstract}

\emph{Sensemaking} is an iterative and dynamic process, consisting of collecting data relevant to the task at hand, organizing and synthesizing the collected information to produce new knowledge and possibly informing further actions. During the sensemaking process, it is difficult for the human working memory to keep track of the progress and all the findings. Moreover, synthesis and management of these findings and all (possibly conflicted) hypotheses emerged increase the cognitive load of the users, thus further impede the sensemaking process. \emph{Analytic provenance} captures both the data exploration process and and its accompanied reasoning, potentially address the information overload and disorientation problems. \emph{Visualization} helps people perform tasks more effectively through visual representations of datasets. The visualization of provenance data could help recall, revisit and reproduce the sensemaking process. More interesting and challenging, analytic provenance has the potential to facilitate the ongoing sensemaking process rather than providing only post hoc support.

This thesis addresses the challenge of how to design interactive visualizations of provenance data for supporting sensemaking. Its original contribution includes the set of four novel interactive visualizations that help users to explore the complex temporal and rational relationship hidden in the sensemaking process through both automatically and manually captured provenance data. First, SchemaLine, a timeline visualization, allows intelligence analysts to construct and refine narratives from their annotations. Second, TimeSets extends SchemaLine to help explore more complex relationship by effectively visualizing both temporal and categorical information simultaneously. Third, SensePath, a visualization tool, captures the provenance of user actions automatically and visualizes it to enable analysts to gain a deep understanding of the user's sensemaking process. Fourth, SenseMap, with a focus on everyday online sensemaking, captures a user's browsing history and visualizes it to provide an overview of the sensemaking process and the relationship between its steps, allowing the user to quickly retrieve relevant information, curate it and consolidate their understanding of the sensemaking problem. All of these four visualizations are developed with a user-centered design approach and evaluated empirically to explore how they help target users make sense of their real tasks. In summary, this thesis contributes novel interactive visualizations of provenance data that enable users to perform effective sensemaking.
\end{abstract}