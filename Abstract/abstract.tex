% ************************** Thesis Abstract *****************************
% Use `abstract' as an option in the document class to print only the titlepage and the abstract.
\begin{abstract}

\emph{Sensemaking} is an iterative and dynamic process, consisting of collecting data relevant to the task at hand, organizing and synthesizing the collected information to produce new knowledge and possibly informing further actions. During the sensemaking process, it is difficult for the human working memory to keep track of the progress and all the findings. Moreover, synthesis and management of these findings and all (possibly conflicted) hypotheses emerged increase the cognitive load of the users, thus further impede the sensemaking process. \emph{Analytic provenance} captures both the data exploration process and and its accompanied reasoning, potentially address the information overload and disorientation problems. \emph{Visualization} helps people perform tasks more effectively through visual representations of datasets. The visualization of provenance data could help recall, revisit and reproduce the sensemaking process. More interesting and challenging, analytic provenance has the potential to facilitate the ongoing sensemaking process rather than providing only post hoc support.

This thesis addresses the challenge of how to design interactive visualizations of provenance data for supporting sensemaking.