% ************************** Thesis Abstract *****************************
% Use `abstract' as an option in the document class to print only the titlepage and the abstract.
\begin{abstract}
\vspace{-0.7in}
\lettrine{S}{ensemaking} is an iterative and dynamic process, in which people collect data relevant to their tasks, analyze the collected information to produce new knowledge, and possibly inform further actions. During the sensemaking process, it is difficult for the human's working memory to keep track of the progress and to synthesize a large number of individual findings and derived hypotheses, thus limits the performance. \emph{Analytic provenance} captures both the data exploration process and and its accompanied reasoning, potentially addresses these information overload and disorientation problems. \emph{Visualization} can help recall, revisit and reproduce the sensemaking process through visual representations of provenance data. More interesting and challenging, analytic provenance has the potential to facilitate the ongoing sensemaking process rather than providing only post hoc support.

This thesis addresses the challenge of how to design interactive visualizations of analytic provenance data to support such an iterative and dynamic sensemaking. Its original contribution includes four visualizations that help users explore complex temporal and reasoning relationships hidden in the sensemaking problems, using both automatically and manually captured provenance. First \emph{SchemaLine}, a timeline visualization, enables users to construct and refine narratives from their annotations. Second, \emph{TimeSets} extends SchemaLine to explore more complex relationships by visualizing both temporal and categorical information simultaneously. Third, \emph{SensePath} captures and visualizes user actions to enable analysts to gain a deep understanding of the user's sensemaking process. Fourth, \emph{SenseMap} visualization prevents users from getting lost, synthesizes new relationship from captured information, and consolidates their understanding of the sensemaking problem. All of these four visualizations are developed using a user-centered design approach and evaluated empirically to explore how they help target users make sense of their real tasks. In summary, this thesis contributes novel and validated interactive visualizations of analytic provenance data that enable users to perform effective sensemaking.
\end{abstract}