\chapter*{Preface}

\section*{What's in This Thesis?}
\lettrine{S}{ensemaking} is an iterative and dynamic process, in which people collect data relevant to their tasks, analyze the collected information to produce new knowledge, and possibly inform further actions. During the sensemaking process, it is difficult for the human's working memory to keep track of the progress and to synthesize a large number of individual findings and derived hypotheses, thus limits the performance. \emph{Analytic provenance} captures both the data exploration process and and its accompanied reasoning, potentially addresses these information overload and disorientation problems. \emph{Visualization} can help recall, revisit and reproduce the sensemaking process through visual representations of provenance data. More interesting and challenging, analytic provenance has the potential to facilitate the ongoing sensemaking process rather than providing only post hoc support.

This thesis addresses the challenge of how to design interactive visualizations of analytic provenance data to support such an iterative and dynamic sensemaking. Its original contribution includes four visualizations that help users explore complex temporal and reasoning relationships hidden in the sensemaking problems, using both automatically and manually captured provenance. First \emph{SchemaLine}, a timeline visualization, enables users to construct and refine narratives from their annotations. Second, \emph{TimeSets} extends SchemaLine to explore more complex relationships by visualizing both temporal and categorical information simultaneously. Third, \emph{SensePath} captures and visualizes user actions to enable analysts to gain a deep understanding of the user's sensemaking process. Fourth, \emph{SenseMap} visualization prevents users from getting lost, synthesizes new relationship from captured information, and consolidates their understanding of the sensemaking problem. All of these four visualizations are developed using a user-centered design approach and evaluated empirically to explore how they help target users make sense of their real tasks. In summary, this thesis contributes novel and validated interactive visualizations of analytic provenance data that enable users to perform effective sensemaking.

\section*{Acknowledgements}
First and foremost, I would like to express my deep and sincere gratitude to my supervisor, Associate Professor Kai Xu, for his warm encouragement and thoughtful guidance. He has always extended his time in crucial stages of my PhD journey. I still remember how crazy my first submission to the IEEE VAST conference was. On $31^{st}$ March, 2013 -- the submission deadline and Easter Sunday by the way -- Kai and I went to the library to work on the paper. We had Vietnamese food for lunch and pizza for dinner, and had to leave the library at midnight because of the holiday. We went to his house and continued working and submitting until the deadline passed at 1am. Then, he drove me home. What else can I ask for from a supervisor? Even though the submission was rejected, the knowledge and experience I learned from him helped me become a better researcher. As a result, at the third and fourth attempts, I published two VAST papers now.

I am indebted to my second supervisor, Professor William Wong, for his valuable advice. As the head of the Interaction Design Center and the principal investigator of an 18-partner EU-funded project, he is extremely busy. However, William always finds time to discuss about my PhD and encourage me to think deeply about the big picture of my work. I also would like to thank him for his financial support through part-time projects and his annual barbecue events. He is a good chief.

I am grateful to Dr Peter Passmore who was the examiner in my Registration and Transfer vivas for his questions and feedback on my work. I would like to thank my colleagues Dr Neesha Kodagoda, Dr Chris Rooney, Dr Rick Walker, Dr Yongjun Zheng, Dr Simon Attfield, Dr Bob Fields, Ashley Wheat and Dr Nallini Selvaraj for their lively discussion, joy and fun in doing research together in the center. I would like to thank Dr Dong-Han Ham, Dr Aidan Slingsby, Dr Jason Dykes, Dr Jo Wood, Dr Derek Stephens, Dr T.J. Jankun-Kelly, Dr Andy Bardill,  Betul Salman and Dr Kate Herd with whom I had a privilege to collaborate in writing papers and to learn from them. I would like to thank all the PhD students that I see everyday, work and play together: Arni, Amar, Pragya, Joshua, Khrisna, Unai, Ali, Ran and many others that I might forget to mention.

I would like to acknowledge Middlesex University, particularly in the award of the research studentship that financially supported me doing this PhD. I would like to thank Professor Balbir Barn, the Deputy Dean of the Faculty of Science and Technology, who waived the tuition fee for my extra writing year.

Last but not least, I would like to give my special thanks to my parents and my small family: my wife Hien and my two daughters -- Lily and Bella. Their patience and love encouraged me to complete this work. To them I dedicate this thesis.

\vspace{1cm}
\begin{flushright}
Phong Hai Nguyen

London, 09/2017
\end{flushright}