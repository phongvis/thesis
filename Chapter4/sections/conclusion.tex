\section{Conclusion and Future Work}

In this paper, we introduced the TimeSets method to visualize set relationships among events in a timeline. Following the proximity and uniform connectedness principles of grouping, TimeSets groups temporal events vertically with colored backgrounds according to their set memberships. Events shared by two sets are visualized using layers with a color gradient background. TimeSets also dynamically adjusts the event labels between three levels of detail to scale with the number of events. The amount of event labels displayed can be traded for ease of following events chronologically using the traceability layout algorithm. The results from the controlled experiment comparing TimeSets to KelpFusion showed that in overall, TimeSets was significantly more accurate and the participants preferred TimeSets for aesthetics and readability.

Currently, duplicated events can only be discovered when mouse hovering. We are investigating a better visual hint without making the visualization too much cluttered. A formal evaluation is needed to study which technique for multi-set memberships, i.e., multiple/concentric circles and multi-colored label background, is the most effective. To improve the visual representation of aggregated events, we will explore and evaluate approaches mentioned in the discussion of the scalability of the layout. Also, we will address the issues identified in the user evaluation, such as the set area not being a reliable indicator of event number and the irritation from bright set colors after a long viewing period. 