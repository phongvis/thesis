\section{Summary}
This chapter introduces TimeSets to enable users to explore complex temporal relationship by effectively representing both temporal and categorical provenance data. It groups temporal events vertically with colored backgrounds according to their set memberships, and uses colored gradient backgrounds for shared ones. Narrative construction was identified as an important user requirement in intelligence analysis, elicited in the previous chapter. Compared to SchemaLine, TimeSets allows analysts to explore and construct more complex, possibly related narratives. It also achieves a higher scalability through visual representations of events at different levels of detail. 

Originally designed as a timeline visualization of user annotations to support sensemaking in intelligence analysis; however, TimeSets shows a much wider application. We demonstrated that it can be used to make sense of publication data and to visualize a number of different types of \emph{events} besides user annotations such as articles, news and tweets. For lower level tasks such as readability, TimeSets was shown to be significantly more accurate than KelpFusison, and the participants preferred TimeSets for aesthetics.

Many aspects can be extended to further improve TimeSets. Currently, duplicated events can only be discovered when mouse hovering. A better visual hint without making the visualization too much cluttered could be useful. We propose different techniques to encode multi-set memberships and to represent aggregated events. However, formal evaluations should be conducted to examine which options are the most effective. Also, it would be beneficial to address the issues identified in the user evaluation, such as the set area not being a reliable indicator of event number and the irritation from bright set colors after a long viewing period. 

SchemaLine and TimeSets allow users to externalize their sensemaking processes, construct and refine complex temporal frames to consolidate their thoughts. After being able to understand how things happened in a particular order, it is essential to understand the rationale driving them happened in that way. The next chapter will investigate how to design visualizations of analytic provenance data to enable users to explore such rational relationship.