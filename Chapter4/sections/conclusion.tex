\section{Summary}
In this chapter, we introduced the TimeSets method that enable users to explore complex temporal relationship by effectively representing both temporal and thematic provenance data. The method is general and can be used to visualize temporal and categorical data in different domains. TimeSets groups temporal events vertically with colored backgrounds according to their set memberships. Events shared by two sets are visualized using layers with a color gradient background. TimeSets also dynamically adjusts the event labels between three levels of detail to scale with the number of events. The amount of event labels displayed can be traded for ease of following events chronologically using the traceability layout algorithm. The results from the controlled experiment comparing TimeSets to KelpFusion showed that in overall, TimeSets was significantly more accurate and the participants preferred TimeSets for aesthetics and readability.

Many directions can be extended to further improve TimeSets. Currently, duplicated events can only be discovered when mouse hovering. A better visual hint without making the visualization too much cluttered could be useful. We propose different techniques to encode multi-set memberships and to represent aggregated events. However, formal evaluations should be conducted to understand which options are more effective. Also, it would be more beneficial to address the issues identified in the user evaluation, such as the set area not being a reliable indicator of event number and the irritation from bright set colors after a long viewing period. 