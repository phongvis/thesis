\section{Summary}
This chapter introduces TimeSets to enable users to explore complex temporal relationship by effectively representing both temporal and thematic provenance data. The method is general and shown to be effective in making sense of data from different domains such as intelligence analysis and publication data. TimeSets groups temporal events vertically with colored backgrounds according to their set memberships, and uses colored gradient backgrounds for shared ones. The results from the controlled experiment comparing TimeSets to KelpFusion showed that in overall, TimeSets was significantly more accurate and the participants preferred TimeSets for aesthetics and readability.

Many directions can be extended to further improve TimeSets. Currently, duplicated events can only be discovered when mouse hovering. A better visual hint without making the visualization too much cluttered could be useful. We propose different techniques to encode multi-set memberships and to represent aggregated events. However, formal evaluations should be conducted to examine which options are the most effective. Also, it would be beneficial to address the issues identified in the user evaluation, such as the set area not being a reliable indicator of event number and the irritation from bright set colors after a long viewing period. 