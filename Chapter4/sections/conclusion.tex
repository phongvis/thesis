\section{Summary}
This chapter introduces TimeSets to enable users to explore complex temporal relationships by effectively representing both temporal and categorical provenance data. It groups temporal events vertically with colored backgrounds according to their set memberships, and uses colored gradient backgrounds for shared ones. Narrative construction was identified as an important user requirement in intelligence analysis, elicited in the previous chapter. Compared to SchemaLine, TimeSets enables analysts to explore and construct more complex, possibly related narratives. It also achieves a higher scalability through visual representations of events at different levels of detail.

TimeSets was designed to support sensemaking in intelligence analysis; however, it shows a much wider application. We demonstrated that it can be used to make sense of publication data and to visualize a number of different types of \emph{events} besides user annotations such as articles, news and tweets. For lower level tasks such as readability, TimeSets was shown to be significantly more accurate than KelpFusison, and the participants preferred TimeSets for aesthetics.

The major limitation of TimeSets is that it only shows intersections between vertically neighboring sets, which only accounts for a small portion of all possible combinations of intersections. The set ordering algorithm to maximize the number of shared events and interaction to reorder sets partially helped address this issue. Future research can focus on increasing the number of visible intersections, prioritizing more important intersections based on some metrics, and providing an overview of all intersections to suggest further exploration. Currently, duplicated events can only be discovered when mouse hovering. A better visual hint without making the visualization too much cluttered could be useful. We propose different techniques to encode multi-set memberships and to represent aggregated events. However, formal evaluations should be conducted to examine which options are the most effective.

SchemaLine and TimeSets allow users to externalize their sensemaking processes, construct and refine complex temporal frames to consolidate their thoughts. After being able to understand how things happened in a particular order, it is essential to understand their rationale. The next chapter will investigate how to design visualizations of analytic provenance data to enable users to explore such reasoning relationship.

