\section{Introduction}
In the previous chapter, SchemaLine is shown to be effective in exploring temporal relationship of intelligence sensemaking by allowing analysts to construct narratives (or schemas) from their annotations (or events). However, it does not allow an event to be part of multiple schemas, which is common in early data exploration. Also, literature in sensemaking theory (\autoref{sub:lr-sensemaking})  suggests the same requirement. Pirolli and Card's model shows that analysts can generate multiple hypotheses from the same information they found. Data--Frame model proposes that multiple frames can be created to account for the same data. Therefore, it is critical for a timeline visualization for sensemaking to effectively show both \emph{temporal} and \emph{set} (for generality) information of events simultaneously.

Back in 1765, one of the oldest documented timelines produced by Joseph Priestley -- the Chart of Biography~\cite{Priestley1765} (\autoref{fig:lr-biography-chart})-- already denotes sets of elements. The timeline includes two thousand famous persons from 1200 BC to 1800 AD classified into six categories based on their most well-known achievement. The timeline is divided into six horizontal bands, one for each category, to visualize the set relations. However, it is clear that an element cannot be part of multiple sets. 

More sophisticated techniques have been designed to visualize multiple-set events. One technique is to assign set membership to a visual channel of element icons such as color hue or shape~\cite{TimeGlider2016}. However, events in the same set are not necessarily located close to each other, making it difficult to follow them chronologically or to have an overview of the distribution of events~\cite{SimileTimeline2009,TimeGlider2016}. Another common approach is to visually connect events in the same set~\cite{Kumar1998}. Such a method can introduce extra edges and crossings, which hamper the readability of the timeline. 

There has been considerable work on set visualization, which commonly uses closed contours as in Venn or Euler diagrams. Texture and color can be used to depict more complex set relations~\cite{Ware2013}. However, these cannot be applied in timelines because the horizontal positions of events are fixed. Techniques that visualize set relations of data items with fixed locations could be good alternatives. To connect same-set elements, Bubble Sets~\cite{Collins2009a} draws an iso-contour surrounding them, LineSets~\cite{Alper2011} uses a B\'{e}zier curve passing through all the elements, and KelpFusion~\cite{Meulemans2013} employs both lines and areas to connect elements. However, simply applying these methods on top of existing timelines could introduce many crossings between text and visual elements of sets that may reduce readability.

Similar to SchemaLine, this chapter also focuses on making sense of temporal relationship in intelligence analysis domain. However, it addresses more complex relationship by effectively displaying both temporal and set information in data. More specifically, we design a novel timeline visualization -- TimeSets -- to

\begin{itemize}
	\item Shows the events within a set over time and their relationships with other sets.
	\item Dynamically adjusts the level of details of each event to suit the amount of information and display estate.
	\item Uses color gradient backgrounds for events belonging to multiple sets and curved set outlines to emphasize its grouping.
\end{itemize}

To demonstrate possible applications of TimeSets, we discuss case studies in different domains: intelligence analysis and publication data. Also, a controlled experiment was conducted to evaluate the effectiveness of TimeSets. The results showed that TimeSets was significantly more accurate than KelpFusion~\cite{Meulemans2013} -- a state-of-the-art set visualization method, and was the preferred choice by the participants for aesthetics.