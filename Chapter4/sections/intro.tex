\section{Introduction}
Timelines have been applied in many domains including personal biographies~\cite{Plaisant1996},  medical records~\cite{Plaisant1998}, music~\cite{Andre2007} and historical events~\cite{Rosenberg2013}. Events in a timeline are commonly categorized into groups or \textit{sets}. For example, academic publications often cover one or many disciplines; similarly, news articles fall into different categories such as politics and sports. Back in 1765, one of the oldest documented timelines produced by Joseph Priestley -- the Chart of Biography~\cite{Priestley1765} -- already denotes grouping of elements. The timeline includes two thousand famous persons from 1200 BC to 1800 AD, who were classified six categories based on their most well-known achievement. The timeline is divided into six horizontal bands, one for each category, to visualize the set relations.

Timeline visualizations use icons to indicate time-point events~\cite{SimileTimeline2009} and horizontal bars for interval ones~\cite{Plaisant1996}. To show set relations, existing methods either color-code the icons or use different shapes~\cite{TimeGlider2016}. The layouts of these methods simply focus on avoiding overlapping between visual elements such as icons and text~\cite{SimileTimeline2009,TimeGlider2016}. As a result, events in the same set are not always located close to each other. This makes it difficult to follow them chronologically or have an overview of the distribution of events in a set. Another common approach is to visually connect events in the same set~\cite{Kumar1998}. Such a method can introduce extra edges and crossings, which hampers the readability of the timeline. 

There has been considerable work on set visualization, which commonly uses closed contours as in Venn or Euler diagrams. Texture and color can be used to depict more complex set relations~\cite{Ware2013}. However, these cannot be applied in timelines because the horizontal positions of events are fixed. Techniques that visualize set relations of data items with fixed locations could be good alternatives. To connect same-set elements, Bubble Sets~\cite{Collins2009a} draws an iso-contour surrounding them, and LineSets~\cite{Alper2011} uses a B\'{e}zier curve passing through all the elements. KelpFusion employs both lines and areas to connect elements, and has been shown to have a significant advantage in readability tasks when compared to Bubble Sets and LineSets~\cite{Meulemans2013}. However, simply applying these methods on top of existing timelines could introduce many crossings between text and visual elements of sets that may reduce readability.

In this chapter, we propose a novel timeline visualization technique -- TimeSets -- to effectively displays both temporal and set information in data. More specifically, TimeSets 

\begin{itemize}
	\item Clearly shows the events within a set over time and their relationships with other sets.
	\item Dynamically adjusts the level of details of each event to suit the amount of information and display estate.
	\item Uses color gradient backgrounds for events belonging to multiple sets and curved set outlines to emphasize its grouping.
\end{itemize}

To show possible applications of TimeSets, we discuss two case studies with publication data and with intelligence analysis. Also, a controlled experiment was conducted to evaluate the effectiveness of TimeSets. The results showed that TimeSets was significantly more accurate than KelpFusion~\cite{Meulemans2013} -- a state-of-the-art set visualization method, and was the preferred choice by the participants for aesthetics.