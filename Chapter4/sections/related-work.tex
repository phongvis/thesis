\section{Related Work}
\label{sub:ts-review}

\subsection{Set Relations in Timelines}
As presented in the Literature Review chapter, \autoref{sub:lr-gestalt}, Gestalt principles are often used to represent set relations among elements. This section discusses the application of those principles in visualizing set relations of events in timelines.

The principle of \textit{similarity} states that objects are perceptually grouped together if they are similar to each other. This principle is extensively applied to show set relations in timelines by using colors and shapes. Time indicators as icons for time-point events~\cite{SimileTimeline2009} and bars for interval events~\cite{Wang2008} are colored according to event set memberships~\cite{SimileTimeline2009}. Different shapes for icons~\cite{TimeGlider2016} and bars~\cite{Plaisant1998} are also used to distinguish set memberships. It is more challenging to represent multiple set memberships. LineSets~\cite{Alper2011} uses concentric circles for icons, where each circle is colored to represent one set.

According to the \textit{proximity} principle, objects that are close together are perceived to be more related than objects that are spaced further apart. In the Chart of Biography~\cite{Priestley1765}, people within a category are placed in a horizontal band, away from people in other categories. LifeLines~\cite{Plaisant1998} splits medical records into different sets, such as \textit{medication} or \textit{diagnosis}, and places them into vertically stacks, which works well if no two sets overlap. Storyline visualizations~\cite{Tanahashi2012,Liu2013} use curved lines to show interactions among characters within the movie timeline. Character lines converge to a bundle if they appear in the same interaction, and diverge when the it ends. Each line can be considered as a set passing through all of its members, and each interaction is a multi-set event. Thus, this method only works for interval events.

Elements tend to be grouped together if they are visually connected. Following this \textit{uniform connectedness} principle, tmViewer~\cite{Kumar1998} links related entities with line segments. Different line colors, thicknesses, and styles were used to distinguish set relations. This method can show events with multiple set memberships by connecting them with multiple edges. However, extra edges and crossings may negatively impact the readability of the timeline.

When similarity and proximity are applied together, the later principle dominates~\cite{Ware2013}. Moreover, uniform connectedness is stronger than proximity~\cite{Palmer1994}. For example, objects with different colors and shapes but located close together are more likely to be perceived as a group, and distant objects but with a closed contour surrounding them also provide a strong sense of grouping. Applying these ideas to visualize set relations for timelines, methods relying on similarity such as colored icons~\cite{Wang2008} are less effective than spatial grouping methods such as LifeLines~\cite{Plaisant1996}. And those, in turn, are less effective than methods using line segments such as tmViewer~\cite{Kumar1998}. 

\subsection{Set Visualizations}
Sets and their relationships can be visualized using Venn~\cite{Ruskey1997} or Euler~\cite{Rodgers2014} diagrams. Simonetto~et~al.~\cite{Simonetto2009} proposed a technique to automatically visualize sets that were previously not possible with Euler diagrams. However, the complex shapes it produces may reduce visualization readability. In their controlled study, Riche and Dwyer~\cite{Riche2010} showed that for complex set intersections, duplications of shared elements resulted in a better performance in readability tasks than a none-duplicated visualization with more complex shapes. 

These methods assume the positions of set elements are not fixed, which reduces their applicability for geo-located or timeline events. Techniques without such constraints include Bubble Sets~\cite{Collins2009a}, LineSets~\cite{Alper2011}, and KelpFusion~\cite{Meulemans2013}. These methods employ the connectedness principle of the Gestalt laws~\cite{Palmer1994} by connecting set elements using extra visual elements. Bubble Sets draws an iso-contour surrounding elements within a set. This iso-contour is filled with a semi-transparent color so that the intersection between sets is shown as an area of blended color. Collins~et~al.~\cite{Collins2009a} provide an example of applying Bubble Sets to a timeline, in which case a force-directed algorithm is used to adjust the vertical positions of elements while the horizontal position along the time axis is fixed. 

LineSets applies a B\'{e}zier curve to connect data items. The curve follows the shortest path passing through all elements in the set. Its study showed that LineSets outperforms Bubble Sets in certain readability tasks ~\cite{Alper2011}. KelpFusion, a hybrid technique, uses lines for data-sparse areas and surfaces for data-dense areas. The results of an evaluation on readability tasks~\cite{Meulemans2013} demonstrated that it outperforms Bubble Sets in both accuracy and completion time, and outperforms LineSets in completion time. There has been no reported attempt to apply LineSets or KelpFusion to timeline visualizations. It is expected that crossings between lines, areas and the event text may reduce the timeline readability.