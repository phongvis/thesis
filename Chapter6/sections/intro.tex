\section{Introduction}
People often get lost while solving complicated tasks using big datasets over long periods of exploration and analysis. They may forget what they have done, fail to find the information they have discovered before, and do not know where to continue. In the World Wide Web context, this problem is known as the \textit{disorientation} problem~\cite{Conklin1987}. One potential solution is to capture and visualize user actions in such a way that can provide an overview of the sensemaking process to the user such as in graphical browser histories~\cite{Ayers1995,Hightower1998,Milic-Frayling2003}. The graphical histories visualize visited web pages and the linking relationships between them to help users quickly see where they are in the page network and to navigate to the pages they want. However, while solving a sensemaking task online, which requires gathering, restructuring and reorganizing lots of information to gain insight, the disorientation problem becomes more severe and difficult to address. They do not just get lost in the hypertext space but also get lost in the task space. They may be unable to answer the following questions. What has been done so far? Where am I in the context of the overall task? What information should I search for next?

% Generally, what we did
In this chapter, we introduce a tool, \emph{SenseMap}, to support \textit{browser-based online sensemaking} through analytic provenance. We targeted this domain because many everyday sensemaking tasks such as travel planning are now performed online~\cite{Russell2008}.
We followed a user-centered, iterative design process to address the problem. First, user behaviors in online sensemaking are elicited through interviews. Then, a simplified sensemaking model based on Pirolli and Card's model~\cite{Pirolli2005} is derived to better represent these behaviors: users iteratively \textit{collect} information sources relevant to the task, \textit{curate} them in a way that makes sense, and finally \textit{communicate} their findings to others. A series of design workshops was followed to derive requirements, discuss designs, implement and test the prototype in an agile setting. SenseMap consists of three linked views. A \emph{browser view} that is a standard web browser with additional sensemaking support and provenance capture. A \emph{history map} that provides an overview of the sensemaking process based on the captured data. And a \emph{knowledge map} that allows users to curate the relevant information. Communication support is provided in all three components.

To explore how SenseMap is used, we conducted a user study in a naturalistic work setting with five participants completing the same sensemaking task related to their daily work activities. Both quantitative data about user activities with SenseMap and qualitative data through semi-structured interviews were collected. All participants found the visual representation and interaction of the tool intuitive to use. Three of them positively engaged with the tool and produced successful sensemaking outcomes.

In summary, SenseMap contributes
\begin{itemize}
\item A user study exploring user behaviors in online sensemaking with existing browser functionality, and a series of workshops followed up to generate requirements and discuss designs.
\item A visualization tool SenseMap supporting browser-based online sensemaking addressing all the derived requirements.
\item A user evaluation exploring how SenseMap is used in a naturalistic work setting and a discussion of insights gained and design lessons learned.
\end{itemize}