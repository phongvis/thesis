\section{Summary}
In this chapter, to address the last research question, we present SenseMap that enables users to externalize and explore complex rational relationship in browser-based online sensemaking. It automatically captures users' sensemaking actions in the browser view and visualizes them in the history map to provide an overview of their sensemaking processes, preventing users from getting lost in the tasks. This enables users to curate the most relevant information into the knowledge map, improving their understanding of the tasks and potentially guide further exploration. At the end, users can communicate their findings using all three views with different levels of detail, including the summary in the knowledge map, the process in the history map, and the raw data in the browser view.

Our evaluation shows that all participants found the visual representation and interaction of the tool intuitive to use. Three of them engaged positively with the tool and produced successful outcomes. It helped them organize information sources, quickly find and navigate to the sources they wanted, and effectively communicate their findings. However, two participants had a negative experience with the tool and were unable to change their practice from sensemaking through collections of browser tabs.

SenseMap shows much potential to provide a powerful approach to online browser-based sensemaking for a wide spectrum of users. In order to meet this potential, it would be useful to improve the following two key areas. First is the need to design more space-efficient visual representations and layouts, together with smarter interaction and feedback between the browser and two maps, allowing users to work on their browsing activities more comfortably. Second is a deeper understanding of how to maximize trust and reassurance of users with the tool, which could provide design guidelines for developing history and knowledge maps.