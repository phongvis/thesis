% ******************************************************************************
% ****************************** Custom Margin *********************************

% Add `custommargin' in the document class options to use this section
% Set {innerside margin / outerside margin / topmargin / bottom margin}  and
% other page dimensions
\ifsetCustomMargin
  \RequirePackage[left=37mm,right=30mm,top=35mm,bottom=30mm]{geometry}
  \setFancyHdr % To apply fancy header after geometry package is loaded
\fi

% Add spaces between paragraphs
%\setlength{\parskip}{0.5em}
% Ragged bottom avoids extra whitespaces between paragraphs
\raggedbottom
% To remove the excess top spacing for enumeration, list and description.
\usepackage{enumitem}
%\setlist[enumerate,itemize,description]{itemsep=0em}

% *****************************************************************************
% ******************* Fonts (like different typewriter fonts etc.)*************

% Add `customfont' in the document class option to use this section

\ifsetCustomFont
  % Set your custom font here and use `customfont' in options. Leave empty to
  % load computer modern font (default LaTeX font).
  %\RequirePackage{helvet}

  % For use with XeLaTeX
  %  \setmainfont[
  %    Path              = ./libertine/opentype/,
  %    Extension         = .otf,
  %    UprightFont = LinLibertine_R,
  %    BoldFont = LinLibertine_RZ, % Linux Libertine O Regular Semibold
  %    ItalicFont = LinLibertine_RI,
  %    BoldItalicFont = LinLibertine_RZI, % Linux Libertine O Regular Semibold Italic
  %  ]
  %  {libertine}
  %  % load font from system font
  %  \newfontfamily\libertinesystemfont{Linux Libertine O}
  
\usepackage{mathpazo}
\renewcommand*\rmdefault{ppl}
\renewcommand*\sfdefault{phv}
\renewcommand*\ttdefault{lmtt}

\fi

% *****************************************************************************
% **************************** Custom Packages ********************************

% ************************* Algorithms and Pseudocode **************************

%\usepackage{algpseudocode}


% ********************Captions and Hyperreferencing / URL **********************

% Captions: This makes captions of figures use a boldfaced small font.
\RequirePackage[bf,font={small,sf}]{caption}
\usepackage{url}
\urlstyle{tt}

% *************************** Graphics and figures *****************************

% Allow XeLatex to include images without specifying extension
\DeclareGraphicsExtensions{.pdf,.png,.jpg}

%\usepackage{rotating}
%\usepackage{wrapfig}

% Uncomment the following two lines to force Latex to place the figure.
% Use [H] when including graphics. Note 'H' instead of 'h'
%\usepackage{float}
%\restylefloat{figure}

% Subcaption package is also available in the sty folder you can use that by
% uncommenting the following line
% This is for people stuck with older versions of texlive
%\usepackage{sty/caption/subcaption}
\usepackage{subcaption}


% ********************************** Tables ************************************
\usepackage{booktabs} % For professional looking tables
\usepackage{multirow}
\setlength{\tabcolsep}{8pt} % a bit more space between columns

%\usepackage{multicol}
%\usepackage{longtable}
\usepackage{tabularx}
\newcolumntype{L}[1]{>{\raggedright\arraybackslash}p{#1}}
\newcolumntype{C}[1]{>{\centering\arraybackslash}p{#1}}
\newcolumntype{R}[1]{>{\raggedleft\arraybackslash}p{#1}}
\newcolumntype{Y}{>{\raggedright\arraybackslash}X}

% *********************************** SI Units *********************************
\usepackage{siunitx} % use this package module for SI units


% ******************************* Line Spacing *********************************

% Choose linespacing as appropriate. Default is one-half line spacing as per the
% University guidelines

% \doublespacing
% \onehalfspacing
% \singlespacing


% ************************ Formatting / Footnote *******************************

% Don't break enumeration (etc.) across pages in an ugly manner (default 10000)
%\clubpenalty=500
%\widowpenalty=500

%\usepackage[perpage]{footmisc} %Range of footnote options


% *****************************************************************************
% *************************** Bibliography  and References ********************

%\usepackage{cleveref} %Referencing without need to explicitly state fig /table

% Add `custombib' in the document class option to use this section
\ifuseCustomBib
   \RequirePackage[square, sort, numbers, authoryear]{natbib} % CustomBib

% If you would like to use biblatex for your reference management, as opposed to the default `natbibpackage` pass the option `custombib` in the document class. Comment out the previous line to make sure you don't load the natbib package. Uncomment the following lines and specify the location of references.bib file

%\RequirePackage[backend=biber, style=numeric-comp, citestyle=numeric, sorting=nty, natbib=true]{biblatex}
%\bibliography{References/references} %Location of references.bib only for biblatex

\fi

% changes the default name `Bibliography` -> `References'
\renewcommand{\bibname}{References}

% ******************************** Roman Pages *********************************
% The romanpages environment set the page numbering to lowercase roman one
% for the contents and figures lists. It also resets
% page-numbering for the remainder of the dissertation (arabic, starting at 1).

\newenvironment{romanpages}{
  \setcounter{page}{1}
  \renewcommand{\thepage}{\roman{page}}}
{\newpage\renewcommand{\thepage}{\arabic{page}}}


% ******************************************************************************
% ************************* User Defined Commands ******************************
% ******************************************************************************

% *********** To change the name of Table of Contents / LOF and LOT ************

\renewcommand{\contentsname}{Contents}
%\renewcommand{\listfigurename}{My List of Figures}
%\renewcommand{\listtablename}{My List of Tables}


% ********************** TOC depth and numbering depth *************************

\setcounter{secnumdepth}{3}
\setcounter{tocdepth}{3}


% ******************************* Nomenclature *********************************

% To change the name of the Nomenclature section, uncomment the following line

%\renewcommand{\nomname}{Symbols}


% ********************************* Appendix ***********************************

% The default value of both \appendixtocname and \appendixpagename is `Appendices'. These names can all be changed via:

%\renewcommand{\appendixtocname}{List of appendices}
%\renewcommand{\appendixname}{Appndx}

% *********************** Configure Draft Mode **********************************

% Uncomment to disable figures in `draftmode'
%\setkeys{Gin}{draft=true}  % set draft to false to enable figures in `draft'

% These options are active only during the draft mode
% Default text is "Draft"
%\SetDraftText{DRAFT}

% Default Watermark location is top. Location (top/bottom)
%\SetDraftWMPosition{bottom}

% Draft Version - default is v1.0
%\SetDraftVersion{v1.1}

% Draft Text grayscale value (should be between 0-black and 1-white)
% Default value is 0.75
%\SetDraftGrayScale{0.8}


% ******************************** Todo Notes **********************************
%% Uncomment the following lines to have todonotes.

%\ifsetDraftClassic
%	\usepackage[colorinlistoftodos]{todonotes}
%	\newcommand{\mynote}[1]{\todo[author=kks32,size=\small,inline,color=green!40]{#1}}
%\else
%	\newcommand{\mynote}[1]{}
%	\newcommand{\listoftodos}{}
%\fi

% Example todo: \mynote{Hey! I have a note}

% Use this package to annotate on specific text
%\usepackage[inline,ignoremode]{Sty/trackchanges}
%\tcignore{\cite}{1}{0}
%\tcignore{\autoref}{1}{0}
%\tcignore{\textbf}{1}{0}
%\tcignore{\textcm}{1}{0}
%\tcignore{\textch}{1}{0}
%\tcignore{\textcb}{1}{0}

% Always include todo notes
\usepackage[colorinlistoftodos,textwidth=2.1cm]{todonotes}
\presetkeys{todonotes}{size=\sffamily,color=yellow,inline}{}
\newcommand{\issue}[1]{\todo[inline,color=red!50]{#1}}
\newcommand{\sidenote}[1]{\todo[size=\sffamily\scriptsize,color=green!50,noinline]{#1}}


% ******************************** Phong's additional changes **********************************
% Format headings
\usepackage{titlesec}  

%\titleformat{\chapter}[display]
%	{\filcenter}
%	{\color{Gray}\Huge Chapter \thechapter}{0pt}
%	{\color{NavyBlue}\Huge\sffamily}
\titleformat{\section}{\sffamily\Large\color{NavyBlue}}{\thesection}{16pt}{}
\titleformat{\subsection}{\sffamily\large\color{NavyBlue}}{\thesubsection}{14pt}{}
\titleformat{\subsubsection}{\sffamily\color{NavyBlue}}{\thesubsubsection}{12pt}{}
\titleformat{\paragraph}[runin]{\sffamily\bfseries\color{NavyBlue}}{\theparagraph}{}{}
\titleformat{\subparagraph}[runin]{\sffamily\itshape\color{NavyBlue}}{\thesubparagraph}{}{}

% fancy chapter
\definecolor{chapbgcolor}{HTML}{DFEDFF}
\definecolor{chapnumcolor}{HTML}{7FB7FF}

\usepackage[Bjornstrup]{fncychap}
\ChNumVar{\fontsize{76}{80}\usefont{T1}{phv}{m}{n}\selectfont}
\ChTitleVar{\raggedleft\color{NavyBlue}\Huge\sffamily}

\makeatletter

\renewcommand\DOCH{%
	\settowidth{\py}{\CNoV\thechapter}
	\addtolength{\py}{-10pt}
	\fboxsep=0pt%
	\colorbox{chapbgcolor}{\rule{0pt}{40pt}\parbox[b]{\textwidth}{\hfill}}%
	\kern-\py\raise20pt%
	\hbox{\color{chapnumcolor}\CNoV\thechapter}\\%
}

\renewcommand\DOTI[1]{%
	\nointerlineskip\raggedright%
	\fboxsep=\myhi%
	\vskip-1ex%
	\colorbox{chapbgcolor}{\parbox[t]{\mylen}{\CTV\FmTi{#1}}}\par\nobreak%
	\vskip 40pt%
}

\renewcommand\DOTIS[1]{%
	\fboxsep=0pt
	\colorbox{chapbgcolor}{\rule{0pt}{40pt}\parbox[b]{\textwidth}{\hfill}}\\%
	\nointerlineskip\raggedright%
	\fboxsep=\myhi%
	\colorbox{chapbgcolor}{\parbox[t]{\mylen}{\CTV\FmTi{#1}}}\par\nobreak%
	\vskip 40pt%
}
\makeatother

% Improve verbatim
\usepackage{listings}
\lstset{
	columns=flexible,
	breaklines=true
}

% continued numbered list
\newcounter{listnum}

% Emphasized text
\newcommand{\strong}[1]{{\textbf{\color{NavyBlue}{#1}}}} 

% check and cross
\usepackage{amssymb}% http://ctan.org/pkg/amssymb
\usepackage{pifont}% http://ctan.org/pkg/pifont
\newcommand{\cmark}{\ding{51}}%
\newcommand{\xmark}{\ding{55}}%

% for the TimeSets citation figure
\definecolor{f2}{HTML}{51A7F9}
\newcommand{\tshierarchy}{\colorbox[HTML]{8dd3c7}{\emph{hierarchy}}}
\newcommand{\tsinteraction}{\colorbox[HTML]{bebada}{\emph{interaction}}}
\newcommand{\tsoverview}{\colorbox[HTML]{FDB462}{\emph{overview}}}
\newcommand{\tsnetwork}{\colorbox[HTML]{FCCDE5}{\emph{network}}}
\newcommand{\tsgraph}{\colorbox[HTML]{80b1d3}{\emph{graph}}}
\newcommand{\tsclustering}{\colorbox[HTML]{B3DE69}{\emph{clustering}}}
\newcommand{\tsevaluation}{\colorbox[HTML]{FFFFb3}{\emph{evaluation}}}

% Image source
\newcommand{\is}[1]{{\textrm{\emph{Image source:~\cite{#1}.}}}} 

% Hyphenation
\hyphenation{SchemaLine}
\hyphenation{TimeSets}
\hyphenation{SensePath}
\hyphenation{SenseMap}

% SenseMap
\definecolor{b1}{HTML}{DE6A10}
\definecolor{b2}{HTML}{0365C0}
\definecolor{b3}{HTML}{00882B}
\newcommand{\textcm}[1]{\textcolor{b1}{#1}}
\newcommand{\textch}[1]{\textcolor{b2}{#1}}
\newcommand{\textcb}[1]{\textcolor{b3}{#1}}

% autoref
\renewcommand{\chapterautorefname}{Chapter}
\renewcommand{\sectionautorefname}{Section}
\renewcommand{\subsectionautorefname}{Section}
\renewcommand{\subsubsectionautorefname}{Section}

% Footnote without marker
\newcommand\blfootnote[1]{%
  \begingroup
  \renewcommand\thefootnote{}\footnote{#1}%
  \addtocounter{footnote}{-1}%
  \endgroup
}