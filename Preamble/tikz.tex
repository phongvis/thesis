% Font
\usepackage{mathpazo}
\usepackage{libertine}
\renewcommand*\ttdefault{lmtt}

\newcommand{\rmt}[1]{{\rmfamily\large{#1}}} 

% Color
\usepackage{xcolor}
\definecolor{f1}{HTML}{F39019}
\definecolor{b1}{HTML}{DE6A10}
\definecolor{f2}{HTML}{51A7F9}
\definecolor{b2}{HTML}{0365C0}
\definecolor{f3}{HTML}{70BF41}
\definecolor{b3}{HTML}{00882B}
\newcommand{\textcm}[1]{\textcolor{b1}{#1}}
\newcommand{\textch}[1]{\textcolor{b2}{#1}}
\newcommand{\textcb}[1]{\textcolor{b3}{#1}}

% tikz
\usepackage{tikz}
\tikzstyle{every node}=[font=\sffamily \large]
\usetikzlibrary{shapes,arrows,positioning,calc,decorations.markings,backgrounds}
\tikzstyle{a1} = [f1!80]
\tikzstyle{c1} = [thick,draw=b1,fill=f1]
\tikzstyle{c2} = [thick,draw=b2,fill=f2]
\tikzstyle{c3} = [thick,draw=b3,fill=f3]
\tikzstyle{cg} = [thick,draw=gray!50,fill=gray!30]
\tikzstyle{rect} = [rectangle, minimum height=1cm]
\tikzstyle{roundrect} = [rect, rounded corners=.2cm]
\tikzstyle{io} = [trapezium, trapezium left angle=70, trapezium right angle=110]
\tikzstyle{arrow} = [thick,->,>=stealth]