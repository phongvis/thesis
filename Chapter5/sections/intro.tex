\section{Background}
Sensemaking is described as the process of comprehension, finding meaning, gaining insight from information, producing new knowledge and informing action (\autoref{sub:lr-sensemaking}). Given the rapid increase in data volume and complexity, more tools are required to support sensemaking, which in many cases remains a slow and laborious process performed by human analysts. The design of such tools requires a deep understanding of the sensemaking process, which is a reoccurring goal of qualitative research conducted by many human-computer interaction (HCI) researchers. Common methods for such qualitative analyses are grounded theory~\cite{Corbin1994} and thematic analysis~\cite{Guest2011}. Typically, researchers need to design a study, collect observation data, transcribe the screen capture videos and think-aloud recordings, identify interesting patterns, group them into categories, and build a model or theory to explain those findings. Unfortunately, this process largely remains manual and thus very time consuming.

In this chapter, we introduce a visual sensemaking tool -- SensePath -- to help HCI researchers recover user's thinking using provenance information. More specifically, we support thematic analysis of online browser-based sensemaking tasks. We chose this domain because many of the everyday sensemaking tasks such as travel planning, are now performed online~\cite{Russell2008}. The design of SensePath is based on the observation of a number of sensemaking sessions and the post hoc analyses that researchers performed to recover the sensemaking process. This is followed by a participatory design session with HCI researchers that led to a number of design requirements such as supporting reasonably long sensemaking tasks (up to two hours), integration with existing qualitative analysis workflow, and non-intrusiveness for participants.

As a result, SensePath was designed to target the \emph{transcription} and \emph{coding} phases during which a researcher needs to transcribe the observation data, such as screen capture video and think-aloud recording (transcription), and then identify the common themes of the sensemaking actions within them and assign appropriate names (coding). SensePath consists of two components designed for different stages of thematic analysis. One runs in the background during the observation to automatically capture provenance data, which includes sensemaking actions. The other component is designed for data analysis and it visualizes the recorded information in four linked views to help transcription, coding, and identify frequent patterns and high level sensemaking process. Two evaluations were conducted to understand how the tool was used by experienced HCI researchers and to discover whether and how SensePath provides any advantages to the analyst compared to traditional analysis methods. The researchers found the tool intuitive and considerably reduces the analysis time, enabling the discovery of underlying sensemaking processes.

%While some features are tailored for sensemaking, the general approach of understanding user's thinking can be applied to a wider qualitative research. Also, SensePath can be extended to support the analysis of other online activities, not limited to sensemaking tasks.

In summary, this chapter contributes
\begin{itemize}
\item A qualitative study and a participatory design session to understand characteristics of qualitative research on sensemaking.
\item A visual sensemaking tool SensePath enabling researchers to explore reasoning relationships of a user's sensemaking process. It supports the transcription and coding of the observation data of online sensemaking tasks
% and can be potentially applied to other qualitative research in HCI beyond sensemaking.
\item A qualitative user evaluation that demonstrated the effectiveness of SensePath.
\end{itemize}