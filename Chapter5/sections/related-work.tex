\section{Related Work on Qualitative Research}
Qualitative research methodologies~\cite{Adams2008} are typically used in study of sensemaking. They allow researchers to reveal complex user experiences and understand issues that are experienced subjectively or collectively~\cite{Pace2004, Adams2008}. Moreover, sensemaking research is often concerned not with testing an existing theory, but building a new one through the collection and analysis of relevant data, generating new knowledge about users and the usage of technology~\cite{Rogers2012}.

Inductive approaches to qualitative research are commonly applied, such as grounded theory~\cite{Corbin1994}, content analysis~\cite{Stemler2001} and thematic analysis~\cite{Guest2011}. These methods rely on the interpretation of rich textual and multimedia data, through manual processing of data and coding before describing it in the context of categories or themes. Moreover, in the case of multimedia data, transcription of audio or video data is often also required. Though these approaches lead to important insight, they are labor intensive, time-consuming and costly in their application~\cite{Wong2002}. Software packages are designed to provide qualitative researchers useful ways to code and index data, and manage the evolving complexity of the process~\cite{Lewins2007}. However, a qualitative analysis is still a largely manual process requiring a substantial investment of time and resources in leading to insightful findings.


