\section{Summary}
In this chapter, we introduce a visual analysis tool SensePath enabling users to explore rational relationship of the sensemaking process. It facilitates thematic analysis of semi-automatic provenance data for online sensemaking tasks, targeting the transcription and coding phases. The data is visualized in four linked views: timeline, browser, replay and transcription. The timeline provides an overview of the sensemaking process and can support a reasonably long sensemaking session common in qualitative research observations. The browser view shows the web page the participant was looking at when performing a sensemaking action, and is complemented by the replay view with the screen capture video of the action. The transcription view provides all the details for a set of actions and can export the information in a format compatible with popular qualitative analysis softwares such as InqScribe, so that analysts can continue working in their existing workflow. 

An evaluation was conducted with an experienced qualitative researcher, who found many features of SensePath helpful for her work, and the data collected from the observation showed that SensePath met most of the requirements it set out to achieve. Another evaluation with two HCI researchers suggested the potential of improvement in completion time for SensePath in the transcription and coding phases.

SensePath can be extended to support exploring rational relationship in other domains beyond online sensemaking tasks. The visualization component can be reused straightforwardly. However, the capture component of SensePath is currently tightly associated with extracting sensemaking actions in a web page, thus needs to be updated. Of course, a discussion with targeted users is required to understand what actions and information are important to capture.  Also, to make this more accessible for non-technical users (such as the analysts) in adding automatic detection of ``search actions'' from new web services, we plan to build a simple GUI, in which they can specify the search templates, thus save the effort of manually modifying the code.