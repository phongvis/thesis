\section{Approach and Requirements}

% How do you go about solving the problem: supporting temporal sensemaking?
% Explain our choice of support: schematization.
We began by exploring the literature to identify the essential points of the sensemaking process to support. As discussed in the Literature Review chapter, \autoref{sub:lr-pcm}, Pirolli and Card describe sensemaking as a cyclic process including two major loops: the foraging loop and the sensemaking loop. In that model, \emph{schematization} serves as a bridge connecting the two loops and plays an important role in converting raw evidence to rational explanations. Pirolli and Card~\cite{Pirolli2005} suggest that schematization should be supported by a computer-based tool that coordinates events in the dataset to reveal their relationships, and to reduce the effort from users in memorizing them. Therefore, we decided to support \emph{temporal schematization} through interactive visualization.

% Transition from schematization to Data-frame model
Also discussed in the Literature Review chapter, \autoref{sub:lr-dfm}, the cognitive processes in schematization are well elaborated through various sensemaking activities centering around \emph{data} and \emph{frame} in the Data--Frame model. For instance, during sensemaking, the user finds some pieces of interesting information. Then, he or she realizes that these pieces mention the same person, thus decides to connect them based on when each event happens to reveal the hidden story. Using the terminology of the Data--Frame model, the user connects data (the pieces of interesting information) to a frame (the story). We decided to enable users to perform all sensemaking activities in the Data--Frame model through intuitive interaction with a timeline visualization. 

% Terminology definition
Note that we use \emph{schema} and \emph{frame} to refer to to the same concept: a structure that defines the relationship of data. In this chapter, we focus on the temporal relationship, explaining the chronology of events discovered during the sensemaking process. We use \emph{events} instead of general data to emphasize on their temporal aspect. Therefore, a schema or frame can be defined as a chronological sequence of related events.

We propose the following design requirements for our timeline visualization.

\begin{enumerate}
	\item \textbf{Automatic layout}. Provide an automatic layout to leverage users from manual arrangement of events.
	\item \textbf{Natural flow}. Easy to follow events within the same schema chronologically.
	\item \textbf{Data--frame model}. Enable users to perform sensemaking activities described in the Data--frame model through intuitive interaction.
\end{enumerate}