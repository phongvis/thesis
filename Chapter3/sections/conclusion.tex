\section{Conclusion and Future Work}
\label{sec:conclusion}

In this paper, we introduced a new timeline visualization, SchemaLine, which is designed to support sensemaking. More specifically, it facilitates the schematization process in the Pirolli-Card model and targets all sensemaking activities in the Data-Frame model. The SchemaLine layout algorithm produces simple, compact, but aesthetically pleasing timeline visualizations. It replaces menu and buttons with fluid user interactions to perform all necessary tasks, and can be integrated within larger visual analytic systems. Our preliminary evaluation suggests that the design of SchemaLine is supportive of sensemaking tasks. It was clearly a helpful aid to users in analysis of the scenario, as evidenced by their usage patterns and feedback. 

As future work, a more formal evaluation would be beneficial -- perhaps even following integration of SchemaLine into a number of different systems, to allow the specific effect to be separated from the rest of the system. In terms of design of the SchemaLine itself, there are a number of improvements that could be added. Shared events between frames could be better visualized (at present, the event is simply duplicated). There are also obvious issues with scalability: while the timeline will scale comparatively well with number of events, it will scale badly with number of frames, since the set of effective qualitative colors is quite small. Other cues such as texture or line style may help with this problem, but to discover this will require further experimentation.