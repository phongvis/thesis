\section{Summary}
This chapter introduces SchemaLine, a novel timeline visualization, enabling users to explore temporal relationship through the annotations they made during their sensemaking processes, focusing on the context of intelligence analysis. SchemaLine supports the schematization process in the Pirolli-Card's sensemaking model and facilitates all sensemaking activities described in the Data--Frame model through fluid interaction. The visual design makes it easy to follow events within a schema, and the algorithm produces a simple, compact, but aesthetically pleasing visualization. 

A user-centered evaluation was conducted to explore how SchemaLine is used in sensemaking of an intelligence analysis task. All participants thought that the tool was intuitive to use and provided necessary support to make sense of the task. They extensively took notes, constructed temporal frames to organize the notes and elaborated the frames when discovering more evidence. They were also confident in presenting and defending their findings using the created frames.

However, SchemaLine suffers from two limitations. First, events that belong to multiple schemas are currently replicated, which is space-inefficient and may confuse users. Second is the scalability issue with the relatively small number of events that SchemaLine can show. The next chapter will present an improved timeline visualization to address these two problems.