\section{Introduction}
Intelligence analysts often need to examine thousands of reports to identify potential threats from particular persons or organizations. This is challenging because of the large number of documents involved and the complex relationship of entities discovered. The limitation of working memory prevents analysts from keeping track of their discoveries simultaneously and manage them effectively. They need to decompose their complex analytical problems, externalize their thoughts, and consolidate them through manipulation of external representations~\cite{Heuer1999}.

Visual analytics systems~\cite{Pioch2006,Wright2006,Stasko2007} have been designed to facilitate intelligence analysis. Automated techniques are applied to leverage analysts from manual investigation of a large document collection. For instance, named-entity recognition techniques~\cite{Nadeau2007} can help identify entities (i.e., persons, organizations and locations), and topic modeling techniques~\cite{Blei2003} can help extract the main themes discussed. These visual analytics systems also allow analysts to externalize their thoughts by note taking. Analysts are then often supported to freely organized their notes in a way that makes sense to them and facilitates their analyses, such as constructing a timeline for representing a story.

% Problem of exisiting timelines in sensemaking
\emph{Timeline} is a simple yet powerful technique to visualize time-oriented data~\cite{Tufte1983}, allowing exploration and identification of temporal patterns and relationships in the data. It displays events along the time axis and position them at the time points at which they occur or the time ranges over which they last~\cite{Plaisant1996}. Timelines have been applied extensively in visualizing both raw data and analysis findings for supporting sensemaking. POLESTAR~\cite{Pioch2006} and HARVEST~\cite{Gotz2006} allow analysts to take notes, define new knowledge, and explore them through a timeline visualization. Jigsaw~\cite{Gorg2013} uses timelines to organize extracted named entities, one for each type. Similarly, nSpace2 Sandbox~\cite{SandboxTimeline2012} provides the creation of multiple sub-timelines for visualizing different types of artifacts. However, these timeline visualizations either lack an automatic layout~\cite{Pioch2006} or use an overly-simplistic linear layout~\cite{SandboxTimeline2012}. As a result, the visualization requires significant effort from analysts to manually arrange data elements, making it difficult to detect temporal patterns.

Sensemaking includes dynamic activities centering around the collected data and its explanation~\cite{Klein2003}. Therefore, to support the dynamic nature of sensemaking, timeline visualizations should allow analysts to create and edit temporal structures interactively. Also, the interaction should be intuitive and fluid~\cite{Elmqvist2011} to prevent analysts from extra cognitive effort and distraction. However, existing timeline visualization techniques are mainly designed for presenting a known story rather than revealing and constructing a hidden one interactively.

In this chapter, we introduce a novel timeline visualization -- SchemaLine -- to address the aforementioned issues. More specifically, SchemaLine contributes
\begin{itemize}
	\item A visual design for an interactive timeline that groups annotations into user-determined schemas.
	\item A compact and aesthetically pleasing timeline layout.
	\item A set of fluid interactions with the timeline to support the sensemaking activities described in the Data--Frame model~\cite{Klein2003}.
\end{itemize}