\section{Introduction}

%SchemaLine contributes
%\begin{itemize}
%	\item A novel design for an interactive timeline that groups notes into schemas determined by the user
%	\item An algorithm to automatically generate a compact and aesthetically pleasing visualization of these schema on the timeline
%	\item A set of fluid interactions with the timeline to support the sensemaking activities defined in the Data-Frame model
%\end{itemize}

% Overview of timeline visualization and the role in sensemaking
\emph{Timeline} is a simple yet powerful technique to visualize time-oriented data~\cite{Tufte1983}, enabling users to explore and identify temporal patterns and relationships. It plots events along the time axis and position them at the time points at which they occur or the time ranges over which they last~\cite{Plaisant1996}. Pirolli and Card suggest the usefulness of timeline visualizations in the schematization process of sensemaking~\cite{Pirolli2005}. A timeline may help to organize collected information based on a chronological aspect, to reveal hidden temporal relationships, and to construct plausible stories. For instance, one can link all pieces of evidence related to a person according to when they happen to build his profile. 

% Importance of presenting information in story order
A timeline does not only reveal the temporal relationships of user findings, but also affects how easily they can be understood. Pennington and Hastie~\cite{Pennington1991} propose that making a decision in a trial is a sensemaking task, in which the juror constructs stories based on evidence from witnesses, exhibits and arguments; and determines the most plausible one. Moreover, the presentation order of evidence has a significant impact on making such a decision. Their study shows that participants were most likely (78\%) to convict when the prosecution evidence was presented in story order and the defense evidence was presented in non-story order; whereas, participants were least likely (31\%) to convict when these sets of evidence were presented in the other way round.

% Problem of exisiting timelines in sensemaking
Timelines have been applied extensively in sensemaking to visualize both raw data and user findings. POLESTAR~\cite{Pioch2006} and HARVEST~\cite{Gotz2006} enable users to take notes, define new knowledge, and visualize them in a timeline. Jigsaw~\cite{Gorg2013} provides automatic extraction of entities such as people, places and organizations, and a timeline to organize them. nSpace2 Sandbox~\cite{SandboxTimeline2012} allows the creation of multiple bands within the timeline to classify different types of artifacts in the system. However, the timeline visualizations in these systems suffer from several drawbacks. They either lack an automatic layout~\cite{Pioch2006} or use an overly-simplistic linear layout~\cite{SandboxTimeline2012}. As a result, the visualization requires significant effort from users to manually arrange the data items, making it difficult for them to detect any temporal patterns of the data.

Sensemaking is a highly iterative process, with each component closely connected to the rest~\cite{Pirolli2005}. It also includes dynamic activities centering around the collected information and its explanation (Data--Frame sensemaking model~\cite{Klein2003}). The implication for timeline visualization is that it needs to support the dynamic nature of sensemaking by enabling analysts to interactively create and edit temporal structures. Also, this should be achieved through intuitive and fluid interaction to prevent extra cognitive effort and distraction from users. However, existing timeline visualization techniques are mainly designed for presenting a known story rather than interactively revealing and constructing a hidden one.

In this chapter, we introduce a novel timeline visualization technique, SchemaLine, to address the aforementioned issues. More specifically, SchemaLine contributes
\begin{itemize}
	\item A visual design for an interactive timeline that groups user annotations into user-determined schemas.
	\item An algorithm to automatically generate a compact and aesthetically pleasing visualization of these schemas on the timeline.
	\item A set of fluid interactions with the timeline to support the sensemaking activities described in the Data--Frame model.
\end{itemize}