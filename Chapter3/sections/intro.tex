\section{Introduction}

\emph{Timeline} is a simple yet powerful technique to visualize time-oriented data~\cite{Tufte1983}, enabling users to explore and identify temporal patterns and relationships in the data. It displays events along the time axis and position them at the time points at which they occur or the time ranges over which they last~\cite{Plaisant1996}. Pirolli and Card~\cite{Pirolli2005} suggest the usefulness of timeline visualizations in the schematization process of sensemaking. A timeline may help  organize collected information based on a chronological order and construct plausible stories. A user study by Kang, Görg and John Stasko~\cite{Kang2011} also shows that timeline is a common choice of participants for organizing related events in an attempt to make sense of them.

% Importance of presenting information in story order
A timeline does not only reveal the temporal relationships of user findings, but also affects how easily they can be understood. A study by Pennington and Hastie~\cite{Pennington1991} suggests that the presentation order of evidence has a significant impact on making decisions in a court trial. The juror constructs stories based on evidence from witnesses, exhibits and arguments before concluding with the most plausible one. The study shows that participants were most likely (78\%) to convict when the prosecution evidence was presented in story order and the defense evidence was presented in non-story order; whereas, participants were least likely (31\%) to convict when these sets of evidence were presented in the other way round.

% Problem of exisiting timelines in sensemaking
Timelines have been applied extensively in visualizing both raw data and analysis findings for supporting sensemaking. POLESTAR~\cite{Pioch2006} and HARVEST~\cite{Gotz2006} allow users to take notes, define new knowledge, and explore them through a timeline visualization. Jigsaw~\cite{Gorg2013} uses timelines to organize extracted named entities such as people, places and organizations. nSpace2 Sandbox~\cite{SandboxTimeline2012} allows the creation of multiple sub-timelines for visualizing different types of artifacts. However, these timeline visualizations either lack an automatic layout~\cite{Pioch2006} or use an overly-simplistic linear layout~\cite{SandboxTimeline2012}. As a result, the visualization requires significant effort from users to manually arrange the data items, making it difficult for them to detect any temporal patterns of the data.

Sensemaking includes dynamic activities centering around the collected data and its explanation~\cite{Klein2003}. To support the dynamic nature of sensemaking, timeline visualization allows analysts to interactively create and edit temporal structures. Also, this should be achieved through intuitive and fluid interaction to prevent extra cognitive effort and distraction from users. However, existing timeline visualization techniques are mainly designed for presenting a known story rather than interactively revealing and constructing a hidden one.

In this chapter, we introduce a novel timeline visualization -- SchemaLine -- to address the aforementioned issues. More specifically, SchemaLine contributes
\begin{itemize}
	\item A visual design for an interactive timeline that groups user annotations into user-determined schemas.
	\item An algorithm to automatically generate a compact and aesthetically pleasing visualization of these schemas on the timeline.
	\item A set of fluid interactions with the timeline to support the sensemaking activities described in the Data--Frame model~\cite{Klein2003}.
\end{itemize}