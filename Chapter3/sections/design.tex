\section{Design Research and Requirements}

right from the beginning, we should know that what is going to be addressed in each chapter and why

should we mention the motivation 

for example, we know that the thesis address the problem by advancing from manual capture - process to automatic capture - rationale

the goal of this chapter is to understand the temporal relationship through user annotations - it's clear at the title

yes you can do that in the introduction - why understanding such process is important.

what should we write in this section?

what do we do to understand the existing process, then elicit requirements to provide support

goal: understand sense

\subsection{Design Research}
SchemaLine is intended to support tasks in sensemaking of temporal data and is influenced by the well-established sensemaking model proposed by Pirolli and Card~\cite{Pirolli2005}. This model organizes the sensemaking process into two loops: the foraging loop, which involves searching, extracting and organizing information; and the sensemaking loop, which involves building schema, creating and testing hypotheses, and presentation. In this model, \emph{schematization} serves as a bridge connecting the foraging loop and the sensemaking loop. It is a crucial step in converting raw evidence to rational explanations. Pirolli and Card suggest that the schematization process should be supported by a computer-based tool that coordinates events in the dataset to reveal relationships between them and to leverage analysts' effort in memorizing them~\cite{Pirolli2005}. As a result, we decided to investigate timeline visualization support for sensemaking. A timeline can not only reveal the temporal relationships among the findings, but also have a considerable impact on how easily they can be understood: when Pennington and Hastie~\cite{Pennington1991} studied the impact of evidence presentation order on juror decision making, they found that information was easier to understand when presented in chronological order and thus had a significant impact on jurors' decisions. 

We find that the cognitive processes in the schematization process are well elaborated through different sensemaking activities in the Data-Frame model proposed by Klein et al.~\cite{Klein2003}. These sensemaking activities are \emph{Connect data to a frame}, \emph{Elaborate a frame}, \emph{Question a frame}, \emph{Preserve a frame}, and \emph{Reframe}. Sensemaking activities begin when a surprise, unexpected event with respect to our prior knowledge appears. The analyst forms an initial account for the unexpected event by connecting some evidence. In the Data-Frame model's terminology, the analyst tries to match some data to create an initial frame. When encountering new data, the analyst can either add it to the frame to elaborate the frame (if it fits to the frame) or remove existing data (if it cannot fit the frame anymore). The analyst starts questioning the frame when they detect inconsistencies between data, or poor quality data in the frame. Then, they need to decide between preserving the frame by looking for more data, or reframing it by comparing it with other frames, or seeking a completely new frame. Because of the various and detailed sensemaking activities surrounding the \textit{frame} in the Data-Frame model, we decide to support all these five activities in our the timeline visualization through fluid user interactions. The terms `schema' and `frame' are used to refer to the same concept throughout this paper.

After discovering a number of relevant events or pieces of evidence, the analyst starts combining them to form a \textit{schema}. A schema is a set of related events that are connected to each other in a certain way. For example, a schema might contain all events about a particular person. Figure~\ref{fig:schema} shows examples of schema. 

\subsection{Requirements}
We propose the following requirements...
\begin{enumerate}
	\item \textbf{Automatic layout}. Provide an automatic layout to leverage users from manually arrange the events.
	\item \textbf{Natural flow}. Easy to follow events within the same schema chronologically.
	\item \textbf{Data--frame activities}. Enable users to perform sensemaking activities described in the Data--frame model through fluid interactions.
\end{enumerate}

